\documentclass[a4paper, zihao=5]{ctexart}

%%%%%% Packages %%%%%%
\usepackage{anyfontsize}
\usepackage{amsmath, amsthm, amssymb, amsfonts}
\usepackage{mathrsfs}
\usepackage[all]{xy}
\usepackage{color}
\usepackage{graphicx}
\usepackage{hyperref}

%%%%% Style Settings %%%%%
% Hyper links
\definecolor{winered}{rgb}{0.5,0,0}   % 酒红色
\definecolor{indigo}{rgb}{0.0, 0.25, 0.42}   % 靛青色
\hypersetup{
    colorlinks=true, 
    linkcolor=indigo,
    urlcolor=indigo,
    filecolor=indigo,
    citecolor=indigo,
    linktoc=all,    % Make the link cover the whole line
    bookmarksnumbered=true,
    bookmarksopen=true
}

% pagestyle
\usepackage{fullpage}
\setlength{\headsep}{5pt}
\linespread{1.35}

% lists
\usepackage[shortlabels, inline]{enumitem}
\setenumerate{nosep, label=(\alph*)}
\setitemize{nosep}

% footnotes & captions
\usepackage[perpage, stable, flushmargin]{footmisc}
\usepackage[font=small, labelfont={bf}]{caption}

%%%%% Theorem Environment & Numbering %%%%%
\numberwithin{equation}{section}
\numberwithin{figure}{section}
\numberwithin{table}{section}
\theoremstyle{definition}
\newtheorem{theorem}{\indent 定理}[section]
\newtheorem{lemma}[theorem]{\indent 引理}
\newtheorem{conjecture}[theorem]{\indent 猜想}
\newtheorem{proposition}[theorem]{\indent 命题}
\newtheorem{corollary}[theorem]{\indent 推论}
\newtheorem{example}[theorem]{\indent 例}
\newtheorem*{definition}{\indent 定义}
\newtheorem*{notation}{\indent 记号}
\newtheorem*{problem}{\indent 问题}
\newtheorem*{remark}{\indent\normalfont\bfseries 注}

%%%%% 汉化 %%%%%
\newcommand{\pftitle}{\indent\normalfont\bfseries}
\renewcommand{\proofname}{\normalfont\bfseries\indent 证明}
% 脚注样式改为圆圈数字
\usepackage{pifont}
\renewcommand{\thefootnote}{\ding{\numexpr171+\value{footnote}}}

%%%%% math commands %%%%%
% Operators
\let\Re\relax\DeclareMathOperator{\Re}{Re}
\let\Im\relax\DeclareMathOperator{\Im}{Im}
\DeclareMathOperator{\dist}{dist}
\DeclareMathOperator{\diam}{diam}
\DeclareMathOperator{\spt}{spt}
\DeclareMathOperator{\sgn}{sgn}
\DeclareMathOperator{\rank}{rank}

% Symbols
\newcommand{\abs}[1]{\left\lvert #1 \right\rvert}
\newcommand{\norm}[1]{\left\lVert #1 \right\rVert}
\newcommand{\nnorm}[1]{\lVert #1 \rVert}
\newcommand{\bk}[2]{\langle #1,\, #2 \rangle}
\newcommand{\floor}[1]{\left\lfloor #1 \right\rfloor}
\newcommand{\set}[1]{\left\{ #1 \right\}}
\renewcommand{\mid}{:\,}
\newcommand{\eval}[2]{\left.{#1}\right\rvert_{#2}}    % evaluation (a vertical line)
\newcommand{\at}{\left.\vphantom{\int}\right\rvert}     % evaluation (a vertical line)
\renewcommand{\d}{\mathop{}\!d}    % d in the integral (with proper spacing)
\newcommand{\dx}{\mathop{}\!dx}
\newcommand{\dy}{\mathop{}\!dy}
\newcommand{\dz}{\mathop{}\!dz}
\newcommand{\dt}{\mathop{}\!dt}
\newcommand{\dd}[2]{\frac{d{#1}}{d{#2}}}
\newcommand{\pd}[2]{\frac{\partial{#1}}{\partial{#2}}}
\newcommand{\ipd}[2]{\partial{#1}/\partial{#2}}    % inline partial derivative
\newcommand{\covin}[1]{\ \text{in}\ {#1}}
\newcommand{\id}{\text{id}}    % identity map

% Alias (short version of existing commands)
\renewcommand{\leq}{\leqslant}
\renewcommand{\geq}{\geqslant}
\newcommand{\f}[2]{\frac{#1}{#2}}
\newcommand{\p}{\partial}
\newcommand{\bm}[1]{\boldsymbol{#1}}
\newcommand{\ol}[1]{\overline{#1}}
\newcommand{\mc}[1]{\mathcal{#1}}
\newcommand{\mb}[1]{\mathbb{#1}}
\newcommand{\ms}[1]{\mathscr{#1}}
\newcommand{\ds}{\displaystyle}
\newcommand{\sm}{\setminus}
\newcommand{\ls}{\lesssim}
\newcommand{\gs}{\gtrsim}
\renewcommand{\emptyset}{\varnothing}
\newcommand{\iso}{\cong}

% Letters
\newcommand{\N}{\mathbb{N}}
\newcommand{\Z}{\mathbb{Z}}
\newcommand{\R}{\mathbb{R}}
\newcommand{\Q}{\mathbb{Q}}
\newcommand{\C}{\mathbb{C}}
\newcommand{\F}{\mathbb{F}}
\newcommand{\T}{\mathbb{T}}
\newcommand{\e}{\varepsilon}
\newcommand{\vp}{\varphi}


\begin{document}

\title{Kakeya 猜想及其相关问题}
\author{黄克元}
\date{2022/05/26}
\maketitle%

\begin{quote}
    \textbf{摘要.} 本文中, 我们简要概述了现有文献中用于解决 Kakeya 问题的几类技术及其进展, 包括组合几何方法、加性组合方法、尺度归纳方法以及代数几何方法. 然后我们讨论了 Kakeya 问题的几种不同的表述方式以及它们之间的关系, 以及 Kakeya 问题的一些基本元素以及相关的基本论证技巧. 随后我们更深入探讨地了组合几何方法在 Kakeya 问题上的应用, 并介绍了 C\'ordoba、Bourgain 和 Wolff 的经典结果及其证明; 我们还介绍了 Dvir 完全解决有限域上 Kakeya 问题的多项式方法. 最后我们简要介绍了 Fourier 分析和 Kakeya 问题的联系.
\end{quote}

\begin{quote}
    \textbf{Abstract.} In this paper, we give a brief overview of several types of techniques and their progress for solving the Kakeya problem in the existing literature, including combinatorial geometric techniques, additive combinatorial techniques, scale-induction techniques, and algebraic geometric techniques. We then discuss several formulations and their relationships, as well as some basic elements of the Kakeya problem and related basic arguments. Then we explore in more depth the application of combinatorial geometry techniques to the Kakeya problem, and introduce C\'ordoba's, Bourgain's and Wolff's classical results and their proofs; we also introduce Dvir's polynomial method for completely solving the Kakeya problem over finite fields. Finally, we briefly introduce the connections between Fourier analysis and the Kakeya problem.
\end{quote}

\section{Kakeya 问题简介}

1917 年, 日本数学家 Soichi Kakeya 提出了 \textbf{Kakeya 转针问题}: 把平面上的单位长度的线段旋转一周, 至少需要扫过多少面积? 恰恰在同一年, 俄罗斯数学家 Besicovitch 在研究一个和 Riemann 积分相关的问题时也考虑了相同的问题. 他引入了如下的概念:

\begin{definition}
    称 $E\subset\R^n$ 为 \textbf{Kakeya 集} (又称 \textbf{Besicovitch 集}), 若 
    \begin{equation*}
        \forall e\in S^{n-1}\ \exists x\in\R^n:\ x+te\in E\ \forall t\in [-1/2, 1/2].
    \end{equation*}
    也就是说, $E$ 中包含了平行于任意方向的单位线段.
\end{definition}

在 1927 年, Besicovitch 给出了 Kakeya 转针问题的回答, 他证明了:

\begin{theorem}[Besicovitch\cite{besicovitch1928kakeya}]
    $\R^n$ 中存在 Lebesgue 测度为零的 Kakeya 集.
\end{theorem}

关于 ``Kakeya 集可以有多小'' 这个问题, 人们进一步提出了 \textbf{Kakeya 集猜想}:

\begin{conjecture}[Kakeya 集猜想]
    $\R^n$ 中 Kakeya 集的 Hausdorff 维数和 Minkowski 维数均为 $n$.
\end{conjecture}

目前 Kakeya 集猜想仅在 $n=2$ 时得到解决\cite{davies1971some}, 其余维数的情形仍然公开.

在 Kakeya 问题的实际研究中, 人们大多尝试证明的是一个比原猜想略微强一些的定量版本的 Kakeya 猜想, 它是用 \textbf{Kakeya 极大算子}来描述的. 这个概念自然地出现在调和分析的许多问题中, 下面给出详细的定义.

\begin{definition}
    设 $a\in\R^n$, $\omega\in S^{n-1}$, $\delta>0$, 定义\textbf{$1\times\delta$ 圆柱体} (tube) $T_\omega^\delta(a)$ 为
    \[
        T_\omega^\delta(a) := \set{x\in\R^n\mid\abs{(x-a)\cdot \omega}\leq 1/2,\ \abs{(x-a)^\perp}\leq \delta},
    \]
    其中 $x^\perp := x - (x\cdot \omega)\omega$. 我们称 $a$ 为圆柱体的\textbf{中心}, $\omega$ 为圆柱体的\textbf{轴向}.
\end{definition}

\begin{definition}[Kakeya 极大算子]
    对 $f\in L^1_{\text{loc}}(\R^n)$ 和 $\delta>0$, 定义 $f_\delta^*: S^{n-1}\to\R$ 为
    \[
        f_\delta^*(\omega) := \sup_{a\in\R^n}\f{1}{\abs{T_\omega^\delta(a)}}\int_{T_\omega^\delta(a)}\abs{f}.
    \]
\end{definition}

\begin{conjecture}[Kakeya 极大算子猜想]
    \label{conj: 极大算子猜想}
    若 $1\leq p\leq n$, $0<\delta\ll 1$, 则
    \begin{equation}
        \label{eq: kakeya-maximal-conjecture-1}
        \norm{f^*_\delta}_{L^p(S^{n-1})}\ls_\e\delta^{-n/p+1-\e}\norm{f}_{L^p(\R^n)}\quad\forall\e>0.
    \end{equation}
\end{conjecture}

\begin{notation}
    在本文中, 我们默认允许所有不等式中隐含的常数依赖于欧式空间的维数 $n$ 以及 $L^p$ 空间的指数; 也就是说, 我们不显式地写出常数对 $n$ 和 $p$ 的依赖. 例如, 式 \eqref{eq: kakeya-maximal-conjecture-1} 中的常数有可能还依赖于 $p$ 和 $n$. 
\end{notation}

事实上 Kakeya 极大算子猜想比 Kakeya 集猜想要更强; 式 \eqref{eq: kakeya-maximal-conjecture-1} 蕴含着 $\R^n$ 中任一 Kakeya 集的 Hausdorff 维数至少是 $p$. (证明见命题 \ref{prop: maximal func and kakeya dim})

当 $p=1$ 时, \eqref{eq: kakeya-maximal-conjecture-1} 是显然的. 故由 Marcinkeiwicz 插值定理可知, 如果某个 $p_0$ 使得 \eqref{eq: kakeya-maximal-conjecture-1} 成立, 则当 $1\leq p\leq p_0$ 时 \eqref{eq: kakeya-maximal-conjecture-1} 都成立. 因而我们的目标是对尽可能大的 $p$ 证明 \eqref{eq: kakeya-maximal-conjecture-1}. 

Kakeya 极大算子猜想至今仅有 $n=2$ 时的情形已经解决\cite{davies1971some}. 当 $n\geq 3$ 时迄今为止的最佳结果如下表所示:

\begin{center}
    \begin{tabular}{c|c|c}
        $n=3$ & $p = 2.5 + \e_0\ (\e_0>0)$ & Katz-Zahl, 2019\cite{katz2019improved}\\
        \hline
        $n=4$ & $p = 3 + (\sqrt{17665}-97)/600\approx 3.059$ & Katz-Zahl, 2020\cite{katz2020kakeya}\\
        \hline
        $n=5$ & $p = 3.6$ & Zahl, 2021\cite{zahl2021new}\\
        \hline
        $n=6$ & $p = 4$ & Wolff, 1995\cite{wolff1995improved}\\
        \hline
        $n\geq 7$ & $p = (2-\sqrt{2})n + \e_n\ (\e_n>0)$ & Zahl, 2021\cite{zahl2021new}
    \end{tabular}
\end{center}

% 由 $\ell^p$ 和 $\ell^{p'}$ 的对偶性, 可以找到 $\set{y_\omega}_{\omega\in\Omega}\subset\R$, 使得

Kakeya 极大算子猜想还有一个对偶的等价版本 (猜想 \ref{conj: 极大算子猜想, 对偶版本}), 它的表述中隐去了 Kakeya 极大算子, 更能够反映 Kakeya 极大算子猜想的几何本质. 我们将在 \ref{subsec: 对偶极大算子猜想} 小节中证明对偶版本和原版本的等价性.

\begin{conjecture}[Kakeya 极大算子猜想, 对偶版本]
    \label{conj: 极大算子猜想, 对偶版本}
    设 $1\leq p\leq n,\ 0<\delta\ll 1$. 若 $\mb{T}=\set{T}$ 是一族轴向 $\delta$-分离的 $1\times\delta$ 圆柱体, 则
    \begin{equation}
        \label{eq: Kakeya maximal dual}
        \norm{\sum_{T\in\mb{T}} \chi_T}_{p'}\ls_\e\delta^{-n/p+1-\e}\left(\sum_{T\in\mb{T}}\abs{T}\right)^{1/p'}\quad\forall\e>0.
    \end{equation}
    其中 $1/p+1/p'=1$.
\end{conjecture}

在式 \eqref{eq: Kakeya maximal dual} 中, 若 $\mb{T}$ 中的圆柱体的重合程度越高, 则左边就越大; 而右边是只和 $\mb{T}$ 中圆柱体的个数有关, 和圆柱体具体摆放的方式无关. 这说明 Kakeya 极大算子猜想本质上是一个组合几何的问题: 它研究的是轴向 $\delta$-分离的 $1\times\delta$ 圆柱体的 ``重合程度'' 最多有多大. 而 $1\times\delta$ 圆柱体可以看成是单位线段的 ``加粗'', 由此当 $\delta$ 很小时可以自然地把 Kakeya 极大算子猜想和 Kakeya 集联系起来; 我们将在 \ref{subsec: 极大算子 vs Kakeya 集} 小节中说明 Kakeya 极大算子猜想蕴含 Kakeya 集猜想.

尽管 Kakeya 猜想尚未被完全解决, 研究者已经发现它和数学中很多其他重要问题有紧密的联系. Kakeya 猜想本身是调和分析中极为重要的问题, 它的解决将对限制性猜想、Bochner-Riesz 猜想、波动方程局部光滑性猜想等调和分析中的核心问题有重大推动作用\cite{tao2001rotating,bourga1991besicovitch}. Kakeya 问题还可以和很多其他领域和问题产生联系, 例如一些加性组合的问题\cite{laba2008harmonic}、解析数论中的 Montgomery 猜想\cite{bourgain1993distribution}、计算机科学中随机数生成的问题\cite{dvir2011kakeya}等.

\section{现有研究方法概述}

Kakeya 猜想 (以及 Kakeya 极大算子猜想) 从形式上来看虽然是几何测度论和调和分析中的问题, 但在其发展历程中却用到来自很多其他数学分支的工具, 这些工具也起到了很好的效果. 接下来我们简要介绍人们目前尝试解决 Kakeya 问题的主要方向.

\subsection{组合几何的方法}

从直观上来说, Kakeya 问题研究的是不同方向的单位线段可以有多大程度的重合. 重合几何 (incidence geometry) 是组合几何的一个分支, 其研究对象正是各种几何元素 (点、线、面) 的重合现象. 因此, 人们开始尝试在 Kakeya 问题中使用组合几何的技巧.\cite{wolff1999recent} 为此需要引入 Kakeya 问题的一个离散版本的简化模型. 

\begin{definition}
    设 $\mb{F}$ 是 $q$ 元有限域. 称 $E\subset\mb{F}^n$ 是 $\F^n$ 中的 \textbf{Kakeya 集}, 若 $E$ 在每个 ``方向'' 上都包含一条 ``直线'', 即
    \[
        \forall e\in\mb{F}^n\sm\set{0}\ \exists a\in\mb{F}^n:\ a+te\in E\ \forall t\in\mb{F}.
    \]
\end{definition}

\begin{conjecture}[有限域上的 Kakeya 猜想]
    \label{conj: 有限域}
    $\F^n$ 中任一 Kakeya 集都满足 $\# E\gs_n q^n$.
\end{conjecture}

下面简单阐述以下在 Kakeya 问题上应用组合几何技巧的想法. 利用 ``两直线至多交于一点'' 这一最基本的观察, 可以证明 $\# E\gs_n q^{(n+1)/2}$. 而欧氏空间 $\R^n$ 中的圆柱体可以看成 $\F^n$ 中 ``直线'' 的 ``加厚''; 两个圆柱体之间重合部分的测度可以被它们轴向的差距来控制, 即
\begin{equation}
    \label{eq: 圆柱体重合部分的测度估计}
    \abs{T_\omega^\delta(a)\cap T_{\omega'}^\delta(b)}\ls\frac{\delta^n}{\abs{\omega-\omega'}+\delta}.
\end{equation}
式 \eqref{eq: 圆柱体重合部分的测度估计} 可以在连续情形代替 ``两直线至多交于一点'', 如此就可以把组合几何的论证推广到欧氏空间中. 按这个想法, Bourgain 在 1991 年证明了 Kakeya 极大算子猜想在 $p=(n+1)/2$ 时的情形\cite{bourga1991besicovitch}. 

利用更复杂的组合几何的技巧可以把有限域上 Kakeya 问题的结果改进到 $\# E\gs_n q^{(n+2)/2}$; Wolff 在 1995 年把这个想法推广到连续情形, 提出了 ``毛刷'' 论证 (hairbrush argument), 进而证明了 Kakeya 极大算子猜想在 $p=(n+2)/2$ 时的情形.\cite{wolff1995improved} 在 2021 年, Katz 和 Zahl 把 Wolff 的想法推广为 ``平面刷'' 论证 (planebrush argument), 进而证明了 Kakeya 极大算子猜想在 $n=4,\ p\approx 3.059$ 时的情形\cite{katz2020kakeya}, 这也是迄今为止四维情形的最优结果. 

\subsection{加性组合的方法}

最早在 Kakeya 问题上应用的加性组合方法是 Bourgain 提出的 ``三切片'' 论证 (three-slices argument)\cite{bourgain1999dimension}. 下面我们简述其想法. 

设 $\F$ 为 $q$ 元有限域, 记 $A=\set{0}\times\F^{n-1}$, $B=\set{1}\times\F^{n-1}$, $C=\set{1/2}\times\F^{n-1}$ (为简单起见不妨设 $q$ 是奇数使得 $1/2$ 有定义). 从直观上来说, 点对 $(a, b)\in A\times B$ 确定的 ``直线'' 的方向和 $a-b$ 的值是一一对应的. 而 $(a+b)\in 2C$, 并且 $C$ 是一个比较小的集合 ($\# C\sim q^{n-1}$), 这就导致 $a-b$ 可取的值不会太多, 因为
\[
    a + b = a' + b' \implies a - b' = a' - b.
\]
由此就可以对 $\F^n$ 中 Kakeya 集的元素个数进行估计. 更详细的讨论可以查阅 Katz 和 Tao 的综述\cite{katz2002recent}.

利用 ``三切片'' 论证配合加性组合中的一些结论, Bourgain 证明了 $\R^n$ 中 Kakeya 集的 Minkowski 维数 $d\geq \frac{n-1}{2-1/13} + 1$\cite{bourgain1999dimension}, 略微改进了他此前的 $d\geq \frac{n+1}{2} = \frac{n-1}{2} + 1$. 此后加性组合的方法仍然有所进展, Katz 和 Tao 证明了 Minkowski 维数 $d\geq \frac{n-1}{\alpha} + 1$, 其中 $\alpha\approx 1.675$ 为多项式 $\alpha^3-4\alpha + 2$ 的最大根\cite{katz2001new}; Bourgain、Katz 和 Tao 证明了 $\R^3$ 中 Minkowki 维数 $d\geq 5/2+\e_0$ ($\e_0>0$),\cite{bourgain2004sum} 这略微改进了 Wolff 的 $d\geq (n+2)/2$. 

\subsection{尺度归纳和热流}

尺度归纳 (induction on scale) 的基本想法是: 把尺度较小的的圆柱体 (比如说 $1\times\delta$ 圆柱体) 填充进尺度较大的圆柱体 (比如说 $1\times\sqrt{\delta}$ 圆柱体) 中, 用这种方式尝试从 Kakeya 极大算子 $f_\delta^*$ 的估计推出 $f_{\sqrt{\delta}}^*$ 的估计; 然后将这个步骤迭代任意多次, 可以把尺度为 $\delta$ 的情形逐步化归到尺度为 $1$. 利用尺度归纳的方法以及组合的技巧, Katz、Laba 和 Tao 证明了 $\R^3$ 中 Kakeya 集的 Minkowski 维数 $d\geq 2.5+10^{-10}$,\cite{katz2000improved} 略微改进了 Wolff 在 $\R^n$ 中 $d\geq (n+2)/2$ 的估计. 这个方法在维数不太大的情形也带来了微小的进展.\cite{laba2001improved}

尺度归纳方法是让圆柱体的尺度按离散的方式变化, 一个改进这个方法的方向就是引入一个连续变换的尺度参数. Bennett、Carbery 和 Tao 提出了一个连续版本的尺度归纳法, 他们考虑了一些集中分布在 $1\times\sqrt{t}$ 圆柱上的 Gaussian 函数, 然后计算这些 Gaussian 函数的某些组合的 $L^p$ 表达式, 最终发现当 $t$ 变化的时候这些 Gaussian 函数的变化方式类似于某种形式的热流. 而热流具有某种单调性, 他们利用这种单调性证明了一个多线性版本的 Kakeya 猜想\cite{bennett2006multilinear}; 随后 Guth 证明了 Bennett--Carbery--Tao 多线性估计的端点情形\cite{guth2010endpoint}.

\begin{theorem}[Bennett--Carbery--Tao]
    若 $\mb{T}_j\ (1\leq j\leq k)$ 是轴向 $\delta$-分离的 $1\times\delta$ 圆柱体的族, ${k}/{(k-1)}\leq q\leq\infty$, 则存在与 $\delta$ 和 $\mb{T}_j$ 无关的常数 $C>0$, 使得
    \begin{equation}
        \label{eq: BCT 多线性估计-1}
        \norm{\prod_{j=1}^k\left(\sum_{T_j\in\mb{T}_j}\chi_{T_j}\right)}_{L^{q/k}(\R^n)}\leq C\prod_{j=1}^k\left(\delta^{n/q}\#\mb{T}_j\right).
    \end{equation} 
\end{theorem}

利用多线性估计进行尺度归纳可以得出 Kakeya 极大算子的估计, 但这种方法得出的估计并不会比 Wolff 的 $d\geq (n+2)/2$ 更好. 此外, 多线性估计 \eqref{eq: BCT 多线性估计-1} 无法再被改进, 这基本就宣告了利用多线性估计进行尺度归纳的方法无法解决 Kakeya 猜想.

\subsection{代数几何和代数拓扑的方法}

代数几何和代数拓扑的方法在 Kakeya 问题上体现出了巨大的威力. 在 2008 年 Dvir 利用代数几何的方法完美解决了有限域上的 Kakeya 猜想\cite{dvir2009size}, 但他的方法依赖于有限域上多项式的性质, 难以应用到欧氏空间中. Guth 尝试发展 Dvir 的多项式方法, 配合代数拓扑中的 ``三明治定理'' 证明了 Bennett--Carbery--Tao 多线性估计的端点情形.\cite{guth2010endpoint} 在 2013 年, Carbery 和 Valdimarsson 利用代数拓扑中的 Borsuk--Ulam 定理给出了另一个证明.\cite{carbery2013endpoint}

不同于 Dvir 用到的有限域上的代数几何, Zahl 在 2021 年用实代数几何中的工具证明了另外一个版本的多线性 Kakeya 极大算子猜想, 由此配合尺度归纳 (induction on scale) 的技巧证明了 Kakeya 极大算子猜想在 $p=(2-\sqrt{2})n + c_n\ (c_n>0)$ 时成立. 这也是迄今为止 Kakeya 极大算子猜想在大多数维数下的最佳结果.\cite{zahl2021new} 由此我们可以看出代数几何的方法所具有的巨大潜力.

\section{Kakeya 问题的若干表述方式}

\subsection{Kakeya 极大算子和 Kakeya 集维数的关系}
\label{subsec: 极大算子 vs Kakeya 集}

下面我们证明, Kakeya 极大算子的估计可以导出 Kakeya 集维数的估计.

\begin{proposition}
    \label{prop: maximal func and kakeya dim}
    若存在 $p, q\in (1, \infty)$ 和 $\alpha\geq 0$ 使得
    \begin{equation}
        \label{eq: Kakeya 算子的估计}
        \norm{f^*_\delta}_q\ls_\e\delta^{-\alpha-\e}\norm{f}_p,\quad\forall\e>0,
    \end{equation}
    那么 $\R^n$ 中 Kakeya 集的 Hausdorff 维数至少是 $n-p\alpha$.
\end{proposition}

\begin{proof}
    设 $K\subset\R^n$ 是 Kakeya 集, 那么我们只需证明对任意 $d<n-p\alpha$ 都有
    \[
        K\subset\bigcup_{j=1}^\infty B(x_j, r_j),\ r_j\leq 1\implies\sum_j r_j^d\gs_d 1.
    \]

    取定 Kakeya 集 $K$ 的开球覆盖 $\set{B(x_j, r_j)}_{j=1}^\infty$ s.t. $r_j\leq 1$. 下面我们估计 $\sum_j r_j^d$ 的下界. 为此, 我们把 $\set{r_j}$ 进行二进制分割, 然后分别求和. 令
    \[
        J_l := \set{j: 2^{-l} < r_j\leq 2^{-(l-1)}},
    \]
    于是
    \[
        \sum_j r_j^d = \sum_l\sum_{j\in J_l}r_j^d\sim\sum_l (\# J_l)2^{-ld}.
    \]

    对任意 $\omega\subset S^{n-1}$, 存在平行于 $\omega$ 的单位线段 $\gamma_\omega\subset K$. 对 $l\in\N$, 令
    \begin{align*}
        \quad
        E_l &:= \bigcup_{j\in J_l} B(x_j, 3r_j),\quad
        K_l := K\cap E_l,\\
        S_l &:= \set{\omega\in S^{n-1}: \abs{\gamma_\omega\cap K_l}_\R\geq\frac{1}{10l^2}},
    \end{align*}
    其中 $\abs{\cdot}_\R$ 表示一维的 Lebesgue 测度. 注意到
    \[
        1=\abs{\gamma_\omega}_\R\leq\sum_l\abs{\gamma_\omega\cap K_l}_\R,\quad\sum_l\frac{1}{10l^2}<1,
    \]
    故由抽屉原理必然有 $S^{n-1}=\bigcup_l S_l$.

    记 $T^{\delta}(\gamma_\omega)$ 为以 $\gamma_\omega$ 为轴、以 $\delta$ 为半径的圆柱体. 那么对任意 $\omega\in S_l$ 有
    \[
        \abs{T^{2^{-l}}(\gamma_\omega)\cap E_l}\gs\frac{1}{l^2}\abs{T^{2^{-l}}(\gamma_\omega)}.
    \]
    由此可知
    \[
        \norm{(\chi_{E_l})^*_{2^{-l}}}_q^q\geq\int_{S_l}\abs{(\chi_{E_l})^*_{2^{-l}}}^q\d\omega\gs\frac{1}{l^{2q}}
        \abs{S_l}_{S^{n-1}}.
    \]
    另一方面, 由 \eqref{eq: Kakeya 算子的估计} 可知
    \[
        \norm{(\chi_{E_l})^*_{2^{-l}}}_q\ls_\e 2^{l\alpha+l\e}\norm{\chi_{E_l}}_p \ls 2^{l\alpha+l\e}\left(\sum_{j\in J_l}r_j^n\right)^{1/p}\sim 2^{l\alpha+l\e}\left(\# J_l\cdot 2^{-nl}\right)^{1/p}.
    \]
    
    由 $S^{n-1}=\bigcup_l S_l$ 可知, 存在 $l$ 使得 $\abs{S_l}\gs 1$. 对此 $l$ 有
    \[
        \frac{1}{l^2}\ls\norm{(\chi_{E_l})^*_{2^{-l}}}_q\ls 2^{l\alpha+l\e}\left(\# J_l\cdot 2^{-nl}\right)^{1/p}.
    \]
    由此, 
    \[
        \sum_j r_j^{n-p\alpha-2p\e}\gs(\# J_l)2^{-l(n-p\alpha-2p\e)}\gs 2^{lp\e}l^{-2p}\gs_\e 1.
    \]
    这就完成了证明.
\end{proof}

\begin{remark}
    从命题 \ref{prop: maximal func and kakeya dim} 可以看出, 如果希望从 Kakeya 极大算子的 $(p, q)$ 型估计导出 Kakeya 集维数的下界估计, 事实上 $q$ 是无关紧要的, 关键是使得 $p$ 尽量大, 并且使得常数尽量更优.
\end{remark}

\subsection{Kakeya 极大算子猜想的对偶版本}
\label{subsec: 对偶极大算子猜想}

首先介绍一个重要的引理, 它使得我们可以把 Kakeya 极大函数的 $L^p$ 范数 ``离散化''.

\begin{lemma}
    \label{lemma: 极大函数 L^p 范数的离散化} 
    设 $\Omega\subset S^{n-1}$ 是极大的 $\delta$-分离集\footnote{度量空间中的某个集合称为是 $\delta$-分离的, 若其中任意两点的距离都 $\geq\delta$.}, $1\leq p<\infty$, 则
    \begin{equation*}
        \norm{f^*_\delta}_p \sim_p \left(\sum_{\omega\in\Omega}\delta^{n-1} f^*_\delta(\omega)^p\right)^{1/p}
    \end{equation*}
\end{lemma}

\begin{proof}
    首先注意到: 若 $\omega, \omega'\in S^{n-1}$ 满足 $\abs{\omega-\omega'}\ls\delta$, 则 $f^*_\delta(\omega)\sim f^*_\delta(\omega')$; 这是因为任一平行于 $\omega$ 的 $1\times\delta$ 柱体可以被 $\ls 1$ 个平行于 $\omega'$ 的 $1\times\delta$ 柱体覆盖.

    由 $\Omega$ 的极大性可知 $S^{n-1}\subset\bigcup_{\omega\in\Omega} B(\omega, \delta)$, 故
    \[
        \norm{f^*_\delta}_p^p\leq\sum_{\omega\in\Omega}\int_{B(\omega, \delta)} f^*_\delta(\omega')^p\d\sigma(\omega') \sim \sum_{\omega\in\Omega}\delta^{n-1} f^*_\delta(\omega)^p.
    \]
    再由 $\set{B(\omega, \delta/2)}_{\omega\in\Omega}$ 两两无交, 可知
    \[
        \norm{f^*_\delta}_p^p\geq\sum_{\omega\in\Omega}\int_{B(\omega, \delta/2)} f^*_\delta(\omega')^p\d\sigma(\omega') \sim \sum_{\omega\in\Omega}\delta^{n-1} f^*_\delta(\omega)^p.
    \]
    这就完成了证明.
\end{proof}

若 $\Omega\subset S^{n-1}$ 是极大的 $\delta$-分离集, 则 $\#\Omega\sim\delta^{1-n}$. 由引理 \ref{lemma: 极大函数 L^p 范数的离散化} 以及幂平均不等式可知,
\[
    \norm{f^*_\delta}_p\gs\sum_{\omega\in\Omega}\delta^{n-1} f^*_\delta(\omega) \gs \sum_{\omega\in\Omega}\int_{T_\omega}\abs{f},
\]
其中 $T_\omega$ 是任意平行于 $\omega$ 的 $1\times\delta$ 圆柱体. 令 $f$ 取遍 $L^p$, 就得到
\begin{equation}
    \label{eq: 对偶版本弱于原版}
    \norm{\sum_{\omega\in\Omega}\chi_{T_\omega}}_{p'}\ls\norm{K_\delta}_{p\to p},
\end{equation}
其中 $K_\delta: f\mapsto f^*_\delta$ 为 Kakeya 极大算子, $1/p' + 1/p = 1$. 由此自然导出如下的猜想.

\begin{conjecture}[Kakeya 极大算子猜想, 对偶版本]
    \label{conj: 极大算子猜想, 轴向极大分离的对偶版本}
    设 $1\leq p\leq n$, $0<\delta\ll 1$. 若 $\T=\set{T}$ 是极大的轴向 $\delta$-分离 $1\times\delta$ 圆柱体的族, 则
    \begin{equation*}
        \norm{\sum_{T\in\T}\chi_T}_{p'}\ls_\e\delta^{-n/p+1-\e}\quad\forall\e>0
    \end{equation*}
    其中 $1/p+1/p'=1$.
\end{conjecture}

由式 \eqref{eq: 对偶版本弱于原版} 可知, Kakeya 极大算子猜想原本的版本 (猜想 \ref{conj: 极大算子猜想}) 蕴含其对偶版本. 下面我们证明, 对偶版本事实上和原来的版本是等价的.

\begin{notation}
    对固定的 $0<\delta\ll 1$ (即 Kakeya 问题中圆柱体的半径), 我们用 $A\lessapprox B$ 或 $B\gtrapprox A$ 表示 $A\ls_\e\delta^{-\e}B\ \forall\e>0$; 用 $A\approx B$ 表示 $A\lessapprox B$ 且 $A\gtrapprox B$.
    
    引进这个记号的原因在于, 证明 Kakeya 极大函数猜想的时我们总可以接受 $\delta^{-\e}$ 的损失, 也就是说 $A\lessapprox B$ 和 $A\ls B$ 基本上是一样的.
\end{notation}

\begin{proposition}[Kakeya 极大算子的对偶估计]
    \label{prop: 对偶版本的等价性}
    设 $1< p\leq n,\, 0<\delta\ll 1$. 那么对 $A\gtrapprox 1$, 下列说法等价:
    \begin{enumerate}[nosep, label=(\alph*)]
        \item $\norm{f^*_\delta}_p\lessapprox A\norm{f}_p$.
        \item 若 $\Omega\subset S^{n-1}$ 为 $\delta$-分离集, 则对任意以 $\omega\in\Omega$ 为轴向的 $1\times\delta$ 圆柱体 $T_\omega$, 有
        \[
            \norm{\sum_{\omega\in\Omega} \chi_{T_\omega}}_{p'}\lessapprox A.
        \]
        \item 若 $\Omega\subset S^{n-1}$ 为 $\delta$-分离集, 则对任意以 $\omega\in\Omega$ 为轴向的 $1\times\delta$ 圆柱体 $T_\omega$, 有
        \[
            \norm{\sum_{\omega\in\Omega} \chi_{T_\omega}}_{p'}\lessapprox A\left(\sum_{\omega\in\Omega}\abs{T_\omega}\right)^{1/p'}.
        \]
    \end{enumerate}
\end{proposition}

\begin{proof}
    由前面的讨论可知 (a) 蕴含 (b). 下面我们分别证明 (c) 蕴含 (a) 以及 (b) 蕴含 (c).

    (I) 首先假设 (c) 成立, 下面证明 (a) 也成立. 
    
    由引理 \ref{lemma: 极大函数 L^p 范数的离散化} 以及 $\ell^p$ 和 $\ell^{p'}$ 的对偶性可知:
    \[
        \norm{f^*_\delta}_p \sim \left(\sum_{\omega\in\Omega}\delta^{n-1}f^*_\delta(\omega)^p\right)^{1/p} = \sum_{\omega\in\Omega}\delta^{n-1} y_\omega f^*_\delta(\omega),
    \]
    其中 $\set{y_\omega}$ 是某些实数, 满足 $\sum_{\omega\in\Omega}\abs{y_\omega}^{p'}=\delta^{1-n}$. 于是存在以 $\omega$ 为轴向的 $1\times\delta$ 柱体 $T_\omega$ 使得
    \[
        \norm{f^*_\delta}_p\sim\sum_{\omega\in\Omega}\delta^{n-1} y_\omega f^*_\delta(\omega)\ls\sum_{\omega\in\Omega}\int_{T_\omega}y_\omega\abs{f}\leq\norm{f}_p\norm{\sum_{\omega\in\Omega}y_\omega\chi_{T_\omega}}_{p'}.
    \]
    下面只需再证明 $\norm{\sum_{\omega\in\Omega}y_\omega\chi_{T_\omega}}_{p'}\lessapprox A$.

    为了把 $\norm{\sum_{\omega\in\Omega}y_\omega\chi_{T_\omega}}_{p'}$ 中的 $\set{y_\omega}$ ``分离'' 出来, 我们采用标准的\emph{二进制分割}的技巧, 即把 $\set{y_\omega}$ 按二的幂次划分为不同的阶, 然后对各阶分别求和. 首先注意到
    \[
        \norm{\sum_{\omega\in\Omega:\, \abs{y_\omega}\leq\delta^{n-1}}y_\omega\chi_{T_\omega}}_{p'}\leq\sum_{\omega\in\Omega}\delta^{n-1}\abs{T_\omega}^{1/p'}\ls\delta^{(n-1)(1-1/p)}\ls 1\lessapprox A,
    \]
    也就是说我们只需再证明
    \[
        \norm{\sum_{\omega\in\Omega:\,\abs{y_\omega}>\delta^{n-1}}y_\omega\chi_{T_\omega}}_{p'}\lessapprox A.
    \]
    为此, 我们把 $\set{\omega: \abs{y_\omega}>\delta^{n-1}}$ 划分为 $O(\log 1/\delta)$ 个集合
    \begin{align*}
        \Omega_k := \set{\omega\in\Omega: 2^{k-1}\leq\abs{ y_\omega}< 2^{k}},
    \end{align*}
    那么,
    \[
        \delta^{1-n}=\sum_{\omega\in\Omega}y_\omega^{p'}\gs\sum_{k\geq 1} 2^{kp'}\#\Omega_k.
    \]
    利用 (c) 以及 H\"older 不等式可知
    \begin{align*}
        \norm{\sum_{\omega\in\Omega:\,\abs{y_\omega}>\delta^{n-1}}y_\omega\chi_{T_\omega}}_{p'}&\ls\sum_k 2^{k}\norm{\sum_{\omega\in\Omega_k}\chi_{T_\omega}}_{p'}\\
        &\lessapprox A\sum_k 2^{k}\left(\#\Omega_k\delta^{n-1}\right)^{1/p'}\\
        &\leq A\delta^{(n-1)/p'}\left(\sum_k 2^{kp'}\#\Omega_k\right)^{1/p'}\left(\sum_k 1\right)^{1/p}\\
        &\ls A\delta^{(n-1)/p'}\delta^{(1-n)/p'}(\log 1/\delta)^{1/p}\\
        &\lessapprox A.
    \end{align*}
    这样就证得 (a) 成立.

    (II) 下面假设 (b) 成立, 下面证明 (c) 也成立.

    注意到 $\Omega$ 为极大 $\delta$-分离集时, (b) 和 (c) 是等价的. 这个观察引出了接下来证明的思路. 令
    \[
        B(N) := \sup_{\Omega, T_\omega:\,N/2<\#\Omega\leq N}\norm{\sum_{\omega\in\Omega}\chi_{T_\omega}}_{p'},
    \]
    我们只需证明 $B(N)\lessapprox A(N\delta^{n-1})^{1/p'}$.
    为此, 我们希望选取正交变换 $U$ 使得 $\Omega\cup U(\Omega)$ 仍然是 $\delta$-分离的, 由此就可以建立 $B(N)$ 和 $B(2N)$ 之间的递推关系. 然后再取 $k$ 充分大使得 $2^kN\sim\delta^{1-n}$, 利用 (b) 得出 $B(2^kN)$ 的估计 (这个估计非常精确), 然后再用递推关系就可以反推出 $B(N)$ 的估计.

    但是一般而言, 选取上述的 $U$ 并不容易, 特别是当 $\#\Omega$ 比较大的时候. 令 $U$ 从正交群 $O(n)$ 中随机选取 (给正交群配上 Haar 测度使其成为概率空间), 则几乎必然 $U(\Omega)\cap\Omega=\emptyset$. 尽管 $U(\Omega)\cup\Omega$ 并不一定仍然是 $\delta$-分离的, 我们希望从中去掉一小部分元素之后使其 $\delta$-分离. 考察非 $\delta$-分离的点的对数
    \[
        A(\Omega, U(\Omega)) := \#\set{(\omega,\omega')\in\Omega\times U(\Omega): \abs{\omega-\omega'}\leq\delta},
    \]
    我们希望选取 $U$ 使得 $A(\Omega, U(\Omega))$ 比较小. 注意到
    \[
        A(\Omega, U(\Omega)) = \sum_{\omega,\omega'\in\Omega}\chi_{B(0, \delta)}(\omega-U(\omega')),
    \]
    于是 $A(\Omega, U(\Omega))$ 的期望
    \begin{align*}
        \int_{O(n)}A(\Omega, U(\Omega))\d U = \sum_{\omega,\omega'\in\Omega}f(\omega,\omega'),\quad
        f(\omega,\omega') := \int_{O(n)}\chi_{B(0,\delta)}(\omega, U(\omega'))\d U.
    \end{align*}
    由 Haar 测度的对称性可知 $f(\omega,\omega')$ 和 $\omega'$ 无关, 从而
    \[
        f(\omega, \omega') = \frac{1}{\abs{S^{n-1}}}\int_{O(n)}\int_{S^{n-1}}\chi_{B(0,\delta)}(\omega, U(\omega'))\d\omega'\d U\sim\int_{O(n)}\delta^{n-1}dU = \delta^{n-1},
    \]
    于是
    \[
        \int_{O(n)}A(\Omega, U(\Omega))\d U \sim\delta^{n-1}\#(\Omega\times\Omega)\leq\delta^{n-1}N^2.
    \]
    由抽屉原理就可以选取 $U\in O(n)$ 使得 $A(\Omega, U(\Omega))\ls\delta^{n-1}N^2$. 

    对任意 $\e>0$, 可以选取 $\delta$-分离集 $\Omega\subset S^{n-1}$ 以及 $1\times\delta$ 圆柱体 $\set{T_\omega}$ 使得
    \[
        \norm{\sum_{\omega\in\Omega}\chi_{T_\omega}}_{p'}\geq B(N)-\e.
    \]
    按上述方式, 选取 $U\in O(n)$ 使得 $A(\Omega, U(\Omega))\ls\delta^{n-1}N^2$. 首先注意到
    \[
        \norm{\sum_{\omega\in\Omega\cup U(\Omega)}\chi_{T_\omega}}_{p'}\geq\left(\norm{\sum_{\omega\in\Omega}\chi_{T_\omega}}_{p'}^{p'} + \norm{\sum_{\omega\in U(\Omega)}\chi_{T_\omega}}_{p'}^{p'}\right)^{1/p'}\geq 2^{1/p'}(B(N)-\e).
    \]
    另一方面, 由 $A(\Omega, U(\Omega))\ls\delta^{n-1}N^2$ 可知, 可以在 $\Omega\cup U(\Omega)$ 挖去一个元素个数 $\ls\delta^{n-1}N^2$ 的集合 $S$ 之后就是 $\delta$-分离集; 并且我们可以把 $S$ 划分为 $\leq c$ 个元素个数为 $\leq\delta^{n-1}N^2$ 的 $\delta$ 分离集 ($c$ 只和 $n$ 有关). 于是
    \[
        \norm{\sum_{\omega\in\Omega\cup U(\Omega)}\chi_{T_\omega}}_{p'}\leq\norm{\sum_{\omega\in(\Omega\cup U(\Omega))\sm S}\chi_{T_\omega}}_{p'} + \norm{\sum_{\omega\in S}\chi_{T_\omega}}_{p'}\leq B(2N) + cB(\delta^{n-1}N^2),
    \]
    令 $\e\to 0$, 就得到递推关系
    \begin{equation}
        \label{eq: random rotation, iterate}
        B(N)\leq 2^{-1/p'}B(2N) + cB(\delta^{n-1}N^2),
    \end{equation}
    其中 $c$ 只和 $p$ 与 $n$ 有关.

    取 $k$ 使得 $2^kN\sim\delta^{1-n}$. 要证 $B(N)\lessapprox A(N\delta^{n-1})^{1/p'}$, 只需证 $b_k := 2^{k/p'}B(2^{-k}\delta^{1-n})$ 满足 $b_k\lessapprox A$. 由 \eqref{eq: random rotation, iterate} 可得到 $\set{b_k}$ 满足的递推式
    \[
        b_k\leq b_{k-1} + c2^{-k/p'}b_{2k}.
    \]
    注意到当 $N$ 充分大 (大到 $S^{n-1}$ 中不存在个数 $\sim N$ 的 $\delta$-分离集) 时 $B(N)=0$; 故存在 $C>0$ 使得当 $k>C\log(1/\delta)$ 时 $b_k=0$. 然后考虑
    \[
        a_k = b_k(1+M2^{-k/p'}),
    \]
    则当 $M>0$ 充分大时可以验证:
    \begin{equation}
        \label{eq: Dual maximal conjecture, iterate argument}
        a_k < a_{k-1} + C2^{-k/p'}\left((a_{2k}-a_k)+(a_{k-1}-a_k)\right).
    \end{equation}
    设 $a_{k_0}$ 为有限个非零的 $\set{a_k}$ 中的最大值, 若 $k_0\geq 1$ 则在 \eqref{eq: Dual maximal conjecture, iterate argument} 中取 $k=k_0$ 即可导出矛盾; 故只能 $k_0=0$. 这就说明了 $a_k\leq a_0 = b_0\lessapprox A$. 再由 $a_k\geq b_k$ 即可完成证明.
\end{proof}

\begin{remark}
    从上述证明过程中可以看出, 命题 \ref{prop: 对偶版本的等价性} 中 (a) 推出 (b) 和 (b) 推出 (c) 都不必在不等式的常数中丢失 $\delta^{-\e}$. 也就是说, 我们有
    \[
        \norm{f^*_\delta}_p\ls A\norm{f}_p\implies\norm{\sum_{\omega\in\Omega}\chi_{T_\omega}}_{p'}\ls A
    \]
    以及
    \[
        \norm{\sum_{\omega\in\Omega}\chi_{T_\omega}}_{p}\ls A\implies\norm{\sum_{\omega\in\Omega}\chi_{T_\omega}}_{p}\ls A\left(\sum_{\omega\in\Omega}\abs{T_\omega}\right)^{1/p},
    \]
    其中 $\Omega\subset S^{n-1}$ 为任意 $\delta$-分离集, $T_\omega$ 是任意平行于 $\omega\in\Omega$ 的 $1\times\delta$ 圆柱体.
\end{remark}

由命题 \ref{prop: 对偶版本的等价性}, 即可说明 Kakeya 极大算子的对偶版本 (猜想 \ref{conj: 极大算子猜想, 对偶版本} 和猜想 \ref{conj: 极大算子猜想, 轴向极大分离的对偶版本}) 和原本的版本 (猜想 \ref{conj: 极大算子猜想}) 是等价的. 但是对偶版本中隐藏了 Kakeya 极大算子, 只需要用圆柱体进行表述, 具有更明确的几何意义.

\subsection{Kakeya 极大算子猜想的染色版本}

下面我们介绍 Kakeya 极大算子猜想的染色版本. 这个版本完全用集合以及测度来描述, 不涉及函数的积分, 更有利于组合几何技巧的应用. 首先介绍染色的概念.

\begin{definition}
    设 $\T$ 是一族 $1\times\delta$ 圆柱体, $0<\lambda\leq 1$. 称 $Y$ 为 $\T$ 的\textbf{$\lambda$-染色} ($\lambda$-shading), 若 $Y$ 是一个映射 $Y: T\in\T\mapsto Y(T)$ 使得 $Y(T)\subset T$, 并且 $\abs{Y(T)}=\lambda\abs{T}$.
\end{definition}

Kakeya 极大算子猜想也可以用染色来刻画:

\begin{proposition}[染色估计]
    \label{prop: Kakeya 极大算子猜想的染色版本}
    设 $0<\delta\ll 1$, $1\leq p\leq n$. 那么下列说法等价:
    \begin{enumerate}[label=(\alph*), nosep]
        \item  Kakeya 极大算子猜想在 $p$ 时成立, i.e. $\norm{f^*_\delta}_p\lessapprox\delta^{-n/p+1}\norm{f}_p$.
        \item 对任意轴向 $\delta$-分离的 $1\times\delta$ 圆柱体族 $\T$ 及其 $\lambda$-染色 $Y$, 都有
        \begin{equation}
            \label{eq: shading conj}
            \abs{\bigcup_{T\in\T} Y(T)}\gtrapprox\lambda^p\delta^{n-p}\left(\delta^{n-1}\#\T\right).
        \end{equation}
        % \item 对任意极大的轴向 $\delta$-分离的 $1\times\delta$ 圆柱体族 $\T$ 及其 $\lambda$-染色 $Y$ ($\delta\ls\lambda\leq 1$), 都有
        % \begin{equation*}
        %     \abs{\bigcup_{T\in\T} Y(T)}\gtrapprox\lambda^p\delta^{n-p}.
        % \end{equation*}
    \end{enumerate}
\end{proposition}

在证明命题 \ref{prop: Kakeya 极大算子猜想的染色版本} 之前, 我们需要一个引理, 它说明 Kakeya 极大算子的弱型估计和强型估计只差 $\delta^{-\e}$ 的常数.

\begin{lemma}[弱型估计]
    \label{lemma: Kakeya weak-type esti.}
    设 $p, q\in [1, \infty]$. 若 Kakeya 极大算子满足弱 $(p, q)$ 型估计
    \[
        \lambda\abs{\set{\omega\in S^{n-1}: (\chi_E)^*_\delta(\omega)>\lambda}}^{1/q}\ls A\abs{E}^{1/p}\quad\forall \lambda>0\ \forall E\subset\R^n;
    \]
    则 $\norm{f^*_\delta}_q\lessapprox A\norm{f}_p$.
\end{lemma}

\begin{proof}
    主要想法是对 Kakeya 极大算子进行插值, 以下大致描述插值的步骤, 而略去具体的计算.

    将弱 $(p, q)$ 型估计和平凡的 $(1, \infty)$ 型估计进行插值, 可得 Kakeya 极大算子的强 $(p_\theta, q_\theta)$ 型估计, 其中
    \[
        \frac{1}{p_\theta} = \frac{1-\theta}{p} + \frac{\theta}{1},\quad \frac{1}{q_\theta} = \frac{1-\theta}{q} + \frac{\theta}{\infty}\quad (0<\theta\leq 1).
    \]
    然后再将 $(p_\theta, q_\theta)$ 型估计和平凡的 $(\infty, \infty)$ 型估计进行插值, 可以得到 $(p, q_\theta')$ 型估计. 经计算可知 $q_\theta'>q$, 而 $S^{n-1}$ 测度有限, 从而可以导出一个依赖于 $\theta$ 的 $(p, q)$ 型估计. 然后令 $\theta\to 0$, 即可得出 Kakeya 极大算子的 $(p, q)$ 型范数 $\lessapprox A$.
\end{proof}

然后我们就可以证明我们原本的命题.

\begin{proof}[\pftitle 命题 \ref{prop: Kakeya 极大算子猜想的染色版本} 的证明]
    假设 (a) 成立. 要证明 (b) 成立, 只需将 (a) 中的 $f$ 取为集合 $\bigcup_{T\in\T} Y(T)$ 的特征函数, 然后利用引理 \ref{lemma: 极大函数 L^p 范数的离散化}. 这里不再赘述细节.

    假设 (b) 成立, 下证 (a) 成立. 由引理 \ref{lemma: Kakeya weak-type esti.}, 我们只需证明弱 $(p, p)$ 型估计; 即对 $E\subset\R^n$, 记 $\Omega_\lambda=\set{\omega\in S^{n-1}: (\chi_E)^*_\delta(\omega)>\lambda}$, 我们需要证明
    \begin{equation}
        \label{eq: shading-to-prove}
        \abs{E}\gtrapprox\delta^p\lambda^{n-p}\abs{\Omega_\lambda}.
    \end{equation}

    % 首先考虑 $\lambda\gs\delta$ 的情形. 
    取 $\Omega_\lambda$ 的极大 $\delta$-分离子集 $\set{\omega_k}_{k=1}^M$, 则 $\abs{\Omega_\lambda}\sim M\delta^{n-1}$. 由 $\Omega_\lambda$ 的定义可知, 存在平行于 $\omega_k$ 的 $1\times\delta$ 圆柱体 $T_k$ 使得 $\abs{T_k\cap E}>\lambda\abs{T_k}$. 由此我们可以定义 $\set{T_k}$ 的 $\lambda$-染色 $Y$ 使得 $Y(T_k)\subset T_k\cap E$. 于是由 (b) 可知
    \[
        \abs{E}\geq\abs{\bigcup_{T\in\T}Y(T)}\gtrapprox\delta^p\lambda^{n-p}(\delta^{n-1}M),
    \]
    这就证得 \eqref{eq: shading-to-prove}.
\end{proof}

\begin{remark}
    (1) 这里不加证明地指出染色估计和 Kakeya 集维数之间的关系:
    \begin{itemize}[nosep]
        \item 若式 \eqref{eq: shading conj} 在 $\lambda=1$ 时成立, 即\footnote{$1$-染色在几乎处处的意义下只有 $Y(T)=T$.}
        \[
            \abs{\bigcup_{T\in\T} T}\gtrapprox\delta^{n-p}(\delta^{n-1}\#\T),
        \]
        则 $\R^n$ 中 Kakeya 集的 Minkowski 维数不小于 $p$.
        \item 若式 \eqref{eq: shading conj} 在 $\lambda\approx 1$ 时成立,  则 $\R^n$ 中 Kakeya 集的 Hausdorff 维数不小于 $p$.
    \end{itemize}

    (2) 事实上在证明式 \eqref{eq: shading conj} 时, 只需证明 $\T$ 为极大的 $\delta$-分离 $1\times\delta$ 圆柱体族时的情形; 这一点可以用类似命题 \ref{prop: 对偶版本的等价性} 的证明中的 ``随机旋转'' 的方法来说明. 我们接下来并不会用到这一点, 限于篇幅这里略去证明.
\end{remark}

\section{Kakeya 问题的组合几何方法}

在 Kakeya 问题的研究进程中, 组合几何的方法在上世纪 70--90 年代占据了主流地位, 这些方法也取得了一些重要的成果. 本节中我们将选取一些由组合几何方法导出的经典结果, 详细介绍和分析其中用到的技术.

\subsection{Kakeya 极大算子的 \texorpdfstring{$L^2$}{L\^{}2} 估计}

本小节中我们将证明二维情形的 Kakeya 极大算子猜想. 下面介绍的方法来自 C\'ordoba\cite{cordoba1977kakeya}, 他的方法依赖于 ``两直线交于一点'' 这一最基本的组合几何事实.

\begin{theorem}[$L^2$ 估计]
    \label{thm: 2-dim kakeya}
    设 $0<\delta\ll 1$. 若 $\T=\set{T}$ 是 $\R^2$ 中一族轴向极大 $\delta$-分离的 $1\times\delta$ 矩形, 则
    \[
        \norm{\sum_T \chi_T}_2\ls (\log 1/\delta)^{1/2}.
    \]
    特别地, Kakeya 极大算子猜想在 $p=2$ 的情形成立.
\end{theorem}

\begin{proof}
    对 $T, T'\in\T$, 记 $\angle(T, T')$ 为 $T$ 和 $T'$ 之间轴向的夹角. 注意到
    \[
        \norm{\sum_T \chi_T}_2^2 = \int \sum_{T, T'}\chi_T\chi_{T'} = \sum_{T, T'}\abs{T\cap T'}\ls\sum_{T, T':\,T\neq T'}\frac{\delta^n}{\angle(T, T')} + O(1),
    \]
    其中 $O(1)$ 来自 $T=T'$ 的项的求和.

    注意到对固定的 $T$, 满足 $\angle(T, T')\sim j\delta$ 的 $T'$ 有 $O(1)$ 个, 这里 $0 < j\ls 1/\delta$ 为整数, 于是
    \[
        \sum_{T, T':\,T\neq T'}\frac{\delta^n}{\angle(T, T')}\ls\sum_T\sum_{0<j\ls 1/\delta}\frac{\delta^n}{j\delta}\ls\#\T\log(1/\delta)\delta^{n-1},
    \]
    再由 $\#\T\sim\delta^{1-n}$ 即可完成证明.
\end{proof}

证明的关键是注意到如下几何上的事实:发挥同等效力的是
\begin{equation}
    \label{eq: Cordoba Geo Esti.}
    \abs{T\cap T'}\ls\frac{\delta^n}{\angle(T, T')}.
\end{equation}
式 \eqref{eq: Cordoba Geo Esti.} 可以看成是 ``两直线交于一点'' 的类比. 为更清楚地看出这一点, 以下给出定理 \ref{thm: 2-dim kakeya} 在有限域中的类比.

\begin{proposition}[有限域情形的 $L^2$ 估计]
    \label{prop: 2-dim finite field}
    设 $\F$ 是 $q$ 元有限域. 若 $E\subset\F^2$ 中包含了 $m$ 条不同方向的直线, 则 $\# E\gs mq$.
\end{proposition}

\begin{proof}
    记这 $m$ 条直线为 $\set{l_i}$, 则由 Cauchy-Schwarz 不等式可知
    \begin{align*}
        mq &= \sum_i\#(E\cap l_i)\\
        &\leq (\# E)^{1/2}\left(\sum_{i, j}\#(l_i\cap l_j)\right)^{1/2}\\
        &\leq (\# E)^{1/2}(mq + m(m-1))^{1/2}\quad\text{(当 $i\neq j$ 时 $\#(l_i\cap l_j)\leq 1$)}\\
        &\ls (\# E)^{1/2}(mq)^{1/2}\quad \text{(由 $m\ls q$)}.
    \end{align*}
    因此 $\# E\gs mq$.
\end{proof}

\subsection{Bourgain 的 bush 论证}

Bourgain 在 1991 年证明了 Kakeya 猜想在 $p=(n+1)/2$ 的情形\cite{bourga1991besicovitch}, 本小节将介绍其证明的主要想法. 为清晰的反映证明所用到的组合本质, 我们首先介绍如何用 Bourgain 的方法证明有限域上对应的问题.

\begin{proposition}
    设 $\F$ 是 $q$ 元有限域 ($q\gg 1$). 若 $K\subset\F^n$ 为 Kakeya 集, 则 $\# K\gs q^{(n+1)/2}$.
\end{proposition}

\begin{proof}
    设 $\mc{L}=\set{l}$ 是包含于 $K$ 的所有不同方向的直线, 则 $\# L\sim q^{n-1}$. 对 $x\in\F^n$, 记 $m(x) = \#\set{i: x\in l_i}$ 为 $x$ 的\emph{重数}. 注意到
    \[
        \sum_{x\in K} m(x) = \sum_{l\in\mc{L}}\#(K\cap l) \sim q\cdot q^{n-1},
    \]
    于是由抽屉原理可找到 $x_0\in K$ 使得
    \[
        m(x_0)\gs\frac{q^n}{\# K}.
    \]

    另一方面, 设 $\set{l_i: 1\leq i\leq m(x_0)}$ 是 $\mc{L}$ 中所有通过 $x_0$ 的直线, 则 $\set{l_i\sm\set{x_0}}$ 两两无交, 从而对 $i$ 求和可以得到:
    \[
        \# K\geq \sum_{i=1}^{m(x_0)}\#(l_i\sm\set{x_0}) = m(x_0)(q-1)\gs\frac{q^{n}}{\# K}\cdot q,
    \]
    由此就得到 $\# K\gs q^{(n+1)/2}$.
\end{proof}

上述证明的依赖于 ``两点确定一条直线'' 这一事实. 由此, 如果 $\mc{L}$ 是通过 $x_0$ 的一束直线, 那么它必然在 $x_0$ 以外的点重数很低. 把圆柱体类比为直线, 就可以在欧氏空间中用上类似的想法. 这个想法最先由 Bourgain 提出\cite{bourga1991besicovitch}. 下面我们叙述如何把这个组合的想法用在欧氏空间中的 Kakeya 问题.

\begin{theorem}[Bourgain]
    \label{thm: Bourgain}
    若 $p=(n+1)/2$, 则 $\norm{f^*_\delta}_p\lessapprox\delta^{-n/p+1}\norm{f}_p$.
\end{theorem}

\begin{proof}
    设 $\T$ 是 $\R^n$ 中一族轴向 $\delta$-分离的 $1\times\delta$ 圆柱体, $Y$ 是其 $\lambda$-染色. 我们只需证明
    \begin{equation}
        \label{eq: bush-to-prove}
        \abs{\bigcup_{T\in\T}Y(T)}\gtrapprox\lambda^{(n+1)/2}\delta^{(n-1)/2}(\delta^{n-1}\#\T).
    \end{equation}
    我们把圆柱体类比为 ``直线'', 然后尝试套用有限域中的方法.

    首先注意到, 当 $\lambda\ls\delta$ 时 \eqref{eq: bush-to-prove} 是平凡的, 因为
    \[
        \abs{\bigcup_{T\in\T}Y(T)}\geq\abs{Y(T)}\sim\lambda\delta^{n-1}\gs\lambda^{(n+1)/2}\delta^{(n-1)/2}(\delta^{n-1}\#\T),
    \]
    最后一步中注意 $\delta^{n-1}\#\T\ls 1$.

    下面我们证明, 存在常数 $C$ 使得 \eqref{eq: bush-to-prove} 在 $C\delta\leq\lambda\leq 1$ 时成立. 与有限域的情形类似, 首先我们尝试找一个点, 使得 $\T$ 中的圆柱体在该点处重数较大. 同样用抽屉原理: 记 $Y=\bigcup_{T\in\T} Y(T)$, 则
    \[
        \int_Y\sum_{T\in\T}\chi_T(x)\dx = \sum_{T\in\T}\abs{Y\cap T} \sim\lambda\delta^{n-1}\#\T,
    \]
    从而可找到 $x\in Y$ 使得
    \[
        \sum_{T\in\T}\chi_T(x)\gs\frac{\lambda\delta^{n-1}\#\T}{\abs{Y}}.
    \]
    记 $\T_x = \set{T\in\T: x\in T}$ 为所有包含 $x$ 的圆柱体, 这样我们就构造出了一束重数较高的圆柱体.

    与有限域的想法相同, $\T_x$ 在远离 $x$ 的地方重数较低, 但这还需要圆柱体之间的夹角不能太小. 取 $\T_x$ 中极大的轴向 $(\delta/\lambda)$-分离的子集 $\T_x'$. 由 $\lambda\geq C\delta$, 当 $C$ 足够大时可使得集族 $\set{T\sm B(x, \lambda/2): T\in\T_x'}$ 中的元素两两无交. 注意到
    \[
        \abs{T\cap B(x, \lambda/2)}\leq\frac{\lambda}{2}\abs{T},
    \]
    于是由 $\abs{Y\cap T}=\lambda\abs{T}$ 可知
    \begin{align*}
        \abs{Y}\geq\sum_{T\in\T_x'}\abs{Y\cap (T\sm B(x, \lambda/2))}
        \geq\sum_{T\in\T_x'}\frac{\lambda}{2}\abs{T}
        \sim\lambda\delta^{n-1}\#\T_x';
    \end{align*}
    而
    \[
        \#\T_x'\sim\#\T_x\lambda^{n-1} = \lambda^{n-1}\sum_{x\in\T}\chi_T(x)\gs \lambda^{n-1}\frac{\lambda\delta^{n-1}\#\T}{\abs{Y}},
    \]
    结合以上两式即可完成证明.
\end{proof}

\subsection{Wolff 的 hairbrush 论证}

Wolff 在 1995 年证明了 Kakeya 极大函数猜想在 $p=(n+2)/2$ 时的情形\cite{wolff1995improved}. 

\begin{theorem}[Wolff]
    \label{thm: Wolff}
    若 $p=(n+2)/2$, 则 $\norm{f^*_\delta}_p\lessapprox\delta^{-n/p+1}\norm{f}_p$.
\end{theorem}

下面我们介绍他的证明思路. 这里我们叙述的证明基于 Tao 在 1997 年提出的 ``双线性约简'' 的技巧\cite{Tao-1998-BilinearApproachRestriction}, 它使得 Wolff 的证明可以被更简洁地叙述.

\subsubsection{有限域的类比}

我们仍然首先在有限域的情形来说明 Wolff 证明中所用到的组合技巧.

\begin{proposition}
    \label{prop: finite field hairbrush}
    设 $\F$ 是 $q$ 元有限域 ($q\gg 1$). 若 $K\subset\F^n$ 为 Kakeya 集, 则 $\# K\gs q^{(n+2)/2}$.
\end{proposition}

\begin{proof}
    设 $\mc{L}=\set{l}$ 是包含于 Kakeya 集 $K$ 的所有不同方向的直线. 对 $x\in K$, 记 $\mc{L}_x=\set{l\in\mc{L}: x\in l}$ 为 $\mc{L}$ 中所有经过 $x$ 的直线的集合.
    
    取 $\mu\gg 1$ 待定. 我们称直线 $l\in\mc{L}$ 是\emph{高重数的}, 若 $l$ 上至少有 $q/2$ 个点 $x$ 满足 $\#\mc{L}_x>\mu$. 我们分别考虑如下两类情况.

    情况 1: 假如 $\mc{L}$ 中不存在高重数的直线. 考虑集合 $K' = \set{x\in K: \#\mc{L}_x\leq\mu}$, 对任意 $l\in\mc{L}$ 都有 $\#(l\cap K)>q/2$. 于是
    \[
        \mu\# K'\geq \sum_{x\in K'}\#\mc{L}_x = \sum_{l\in\mc{L}}\#(l\cap K')\geq (q/2)\#\mc{L}\sim q^n,
    \]
    从而 $\#K\geq \# K'\gs \mu^{-1}q^n$.

    情况 2: $\mc{L}$ 中存在高重数的直线 $l_0$. 假设 $x_1, \cdots, x_k\in l_0\ (k\geq q/2)$ 满足 $\#\mc{L}_{x_i}>\mu$. 接下来我们希望选取一些平面来应用命题 \ref{prop: 2-dim finite field}. 记 $ \mc{H}=\bigcup_{i=1}^k(\mc{L}_{x_i}\sm\set{l_0})$ 为所有和 $l_0$ 相交的直线的集合; 再记 $\Pi$ 为所有由 $l_0$ 和 $\mc{H}$ 中某条直线所确定的平面的集合.
    注意到集族 $\set{\pi\sm{l_0}:\pi\in\Pi}$ 中的元素两两无交, 于是
    \[
        \# K\geq \sum_{\pi\in\Pi}\#(K\cap (\pi\sm l_0)).
    \]
    记 $m(\pi)$ 为平面 $\pi$ 中包含 $\mc{H}$ 中直线的条数, 则由二维的结论 (命题 \ref{prop: 2-dim finite field}) 可知
    \[
        \#(K\cap (\pi\sm l_0))\gs m(\pi)q.
    \]
    于是
    \begin{equation}
        \label{eq: discrete hairbrush estimate}
        \# K \gs\sum_{\pi\in\Pi} m(\pi)q = q\#\mc{H}\geq qk\mu\gs q^2\mu
    \end{equation}
    取 $\mu\sim q^{(n-2)/2}$ 即可使情况 1 和 2 都满足 $\# K\gs q^{(n+2)/2}$.
\end{proof}

上面证明中的情况 2 中, 我们本质上是考察了与直线 $l_0$ 相交的一族直线 $\mc{H}$, 这个结构看起来像一柄刷子 (hairbrush). 我们本质上是利用二维的估计, 来说明刷子上的 ``刷毛'' $\mc{H}$ 在远离 ``刷柄'' $l_0$ 的地方重合度不会太高. 式 \eqref{eq: discrete hairbrush estimate} 说明, 当刷毛越多 (即 $\#\mc{H}$ 越大), 我们得出的估计就越好. 所以, 证明的关键在于构造一把有很多毛的刷子, 即找到一条重数很高的直线 $l_0$ 作为刷柄.

\subsubsection{双线性约简}

下面我们考虑如何把上述的组合技巧应用到 $\R^n$ 中. 按照前面的讨论, 我们希望利用某些方法构造出一个重数很高的圆柱体 $T_0$ 以及一个对应的 ``刷子'' , 把刷柄的一个邻域挖去之后刷子剩余的部分重叠部分比较小, 由此导出染色面积的估计. Bourgain 定理 (定理 \ref{thm: Bourgain}) 的证明告诉我们, 要想使得两个圆柱体的两端重叠程度比较低, 那么它们的夹角应该尽量大. 以上这些就启发了\textbf{双线性约简} (bilinear reduction) 的技巧, 它使得我们可以把问题简化到圆柱体的夹角不太小的情况.

% \begin{figure}
%     \centering
%     \includegraphics[scale=0.25]{hairbrush.jpg}
%     \caption{Wolff 的 ``刷子''}
%     \label{fig: hairbrush}
% \end{figure}

我们从 Kakeya 极大算子的对偶版本出发. 设 $\T$ 是一族轴向 $\delta$-分离的 $1\times\delta$ 圆柱体, $p\geq 2$. 我们希望证明
\begin{equation}
    \label{eq: hairbrush to prove 1}
    \norm{\sum_{T\in\T}\chi_T}_{p'}\lessapprox\delta^{-n/p+1}.
\end{equation}
由命题 \ref{conj: 极大算子猜想, 对偶版本}, 我们可以不妨假设 $T$ 的轴向都落在 $B(e_n, 1/10)$ 中, 其中 $e_n$ 是 $\R^n$ 的第 $n$ 个标准基向量; 因为我们只需要把球冠 $B(e_n, 1/10)$ 旋转若干次, 进而覆盖整个球面. 

记 $q=p'$. 把式 \eqref{eq: hairbrush to prove 1} 改写为
\begin{equation}
    \label{eq: bilinear 1}
    \norm{\sum_{T, T'\in\T}\chi_T\chi_{T'}}_{q/2}^{q/2}\lessapprox\delta^{n-q(n-1)}.
\end{equation}
由于 $q/2\leq 1$, 从而有伪三角不等式
\[
    \norm{f+g}_{q/2}^{q/2}\leq\norm{f}_{q/2}^{q/2}+\norm{g}_{q/2}^{q/2}.
\]
对 \eqref{eq: bilinear 1} 的左边按圆柱体的夹角 $\angle(T, T')=\abs{\omega_T-\omega_{T'}}$ 进行二进制分割, 可得
\[
    \norm{\sum_{T, T'\in\T}\chi_T\chi_{T'}}_{q/2}^{q/2}\leq\sum_{k=0}^{O(\log 1/\delta)}\norm{\sum_{T, T': \angle(T, T')\sim 2^{-k}}\chi_T\chi_{T'}}_{q/2}^{q/2} + \norm{\sum_{T, T': T=T'}\chi_T\chi_{T'}}_{q/2}^{q/2}.
\]
容易证明 $T=T'$ 的一项可以被式 \eqref{eq: bilinear 1} 的右边吸收掉; 再由 $O(\log 1/\delta)\lessapprox 1$, 要证 \eqref{eq: hairbrush to prove} 就只需证对任意 $k$ 都有
\begin{equation}
    \label{eq: bilinear 2}
    \norm{\sum_{T, T': \angle(T, T')\sim 2^{-k}}\chi_T\chi_{T'}}_{q/2}^{q/2}\lessapprox\delta^{n-q(n-1)}.
\end{equation}

接着我们尝试对 \eqref{eq: bilinear 2} 做尺度变换, 希望使得参与求和的圆柱体的夹角和 $k$ 无关. 首先把 $S^{n-1}$ 分割为若干个半径为 $2^{-k}/10$ 的球冠 $C_i$, 记 $\T_i\subset\T$ 为轴向落在 $C_i$ 中的圆柱体的子集. 于是要证 \eqref{eq: bilinear 2}, 只需证对每个 $i$ 都有
\begin{equation}
    \label{eq: bilinear 2.5}
    \norm{\sum_{T, T'\in\T_i: \angle(T, T')\sim 2^{-k}}\chi_T\chi_{T'}}_{q/2}^{q/2}\lessapprox\delta^{n-q(n-1)}.
\end{equation}  
不妨设 $C_i$ 的中心是 $e_n$, 然后考虑前 $n-1$ 个坐标的伸缩变换
\[
    L(\underline{x}, x_n) = (2^{k}\underline{x}, x_n),
\]
则式 \eqref{eq: bilinear 2} 等价于
\begin{equation}
    \label{eq: bilinear 3}
    \norm{\sum_{T, T'\in\T_i: \angle(T, T')\sim 2^{-k}}\chi_{L(T)}\chi_{L(T')}}_{q/2}^{q/2}\lessapprox 2^{-k(n-1)}\delta^{n-q(n-1)}.
\end{equation}
大致可以认为 $L(T)$ 是 $1\times 2^k\delta$ 圆柱体, 其轴向认为是落在 $2^kC_i$ 上, 于是当 $T, T'\in\T_i$ 时
\[
    \angle(L(T), L(T')) = \abs{\omega_{L(T)}-\omega_{L(T')}}\sim 2^k\abs{\omega_T-\omega_{T'}}\sim 1.
\]
基于此观察, 如果用 $2^k\delta$ 代替 $\delta$, $2^k C_i$ 代替 $C_i$, 注意到 \eqref{eq: bilinear 3} 右边 $\lessapprox (2^k\delta)^{n-q(n-1)}$, 就会发现 \eqref{eq: bilinear 3} 被 \eqref{eq: bilinear 2.5} 在 $k=0$ 的情形所蕴含. 由此问题又可以化归为证明
\begin{equation}
    \label{eq: bilinear 4}
    \norm{\sum_{T, T': \angle(T, T')\sim 1}\chi_T\chi_{T'}}_{q/2}^{q/2}\lessapprox\delta^{n-q(n-1)},
\end{equation}
其中 $T, T'$ 的轴向都落在某个半径为 $1/10$ 的球冠内. 由此, 我们就只需要考虑夹角近似是一个定值的圆柱体, 从而忽略那些夹角很小的情形.

另外注意到, 要证明 \eqref{eq: bilinear 4}, 只需证
\begin{equation}
    \label{eq: bilinear}
    \norm{\left(\sum_{T\in\T}\chi_T\right)\left(\sum_{T'\in\T'}\chi_{T'}\right)}_{q/2}^{q/2}\lessapprox\delta^{n-q(n-1)},
\end{equation}
其中 $\T$ 和 $\T'$ 是两族轴向 $\delta$-分离的 $1\times\delta$ 圆柱体, 它们的轴向都落在 $B(e_n, 1/10)$ 内, 并且任意 $T\in\T$ 和 $T'\in\T'$ 都满足 $\angle(T, T')\sim 1$. 式 \eqref{eq: bilinear} 的好处在于, 其中包含了体现 ``重数'' 的 $\sum\chi_T$, 这有利于组合技巧的应用. 

\subsubsection{``刷子'' 的构造}

下面尝试对 \eqref{eq: bilinear} 施展二进制分割的技巧. 对 $\mu, \mu'\in 2^{\Z}$ ($=\set{2^s: s\in\Z}$), 定义
\[
    E_{\mu, \mu'} = \set{x\in\R^n: \sum_{T\in\T}\chi_T(x)\sim\mu, \sum_{T\in\T'}\chi_{T'}(x)\sim\mu'}.
\]
于是
\[
    \norm{\left(\sum_{T\in\T}\chi_T\right)\left(\sum_{T'\in\T'}\chi_{T'}\right)}_{q/2}^{q/2}\sim \sum_{\mu, \mu'}(\mu\mu')^{q/2}\abs{E_{\mu, \mu'}}.
\]
由于 $\mu$ 和 $\mu'$ 的取值范围是从 $1$ 到 $O(\delta^{1-n})$, 从而 $\mu$ 和 $\mu'$ 都只有 $O(\log 1/\delta)$ 种取值, 而 $O(\log 1/\delta)\lessapprox 1$, 故要证 \eqref{eq: bilinear} 就只需证每对 $\mu, \mu'$ 都满足
\begin{equation}
    \label{eq: Dyadic partition, mu}
    (\mu\mu')^{q/2}\abs{E_{\mu, \mu'}}\lessapprox\delta^{n-q(n-1)}.
\end{equation}
不失一般性, 我们还可以假设 $\abs{E_{\mu, \mu'}}\geq\delta^{10n}$ (否则 \eqref{eq: Dyadic partition, mu} 是平凡的).

取定 $\mu, \mu'$. 为估计 $\abs{E_{\mu, \mu'}}$, 注意到
\begin{equation}
    \label{eq: E_mu representation}
    \mu\mu'\abs{E_{\mu, \mu'}} \sim \sum_{T\in\T}\sum_{T'\in\T'}\int_{E_{\mu, \mu'}}\chi_T\chi_{T'}.
\end{equation}
对上面式子的右边再次二进制分割. 对 $\lambda, \lambda'\in 2^{\Z}$, 令
\begin{align*}
    \T_\lambda &= \set{T\in\T: \abs{T\cap E}\sim\lambda\abs{T}},\\
    \T_\lambda' &= \set{T'\in\T': \abs{T'\cap E}\sim\lambda\abs{T'}}.
\end{align*}
不难证明 \eqref{eq: E_mu representation} 右边的求和中, $\lambda<\delta^{20n}$ 或 $\lambda'<\delta^{20n}$ 的部分很小, 可以被 \eqref{eq: E_mu representation} 的左端吸收掉 (利用 $\abs{E_{\mu, \mu'}}\geq\delta^{10n}$), 于是
\[
    \mu\mu'\abs{E_{\mu, \mu'}} \sim \sum_{\lambda\geq\delta^{20n}}\sum_{\lambda'\geq\delta^{20n}}\sum_{T\in\T_\lambda}\sum_{T'\in\T'_{\lambda'}}\int_{E_{\mu, \mu'}}\chi_T\chi_{T'}.
\]
其中 $\delta^{20n}\leq\lambda, \lambda'\leq 1$, 故最多只有 $O(\log 1/\delta)$ 项, 故由抽屉原理, 必然存在某一对 $\lambda, \lambda'$ 使得
\begin{equation}
    \label{eq: lambda exist}
    \mu\mu'\abs{E_{\mu, \mu'}}\approx\sum_{T\in\T_\lambda}\sum_{T'\in\T'_{\lambda'}}\int_{E_{\mu, \mu'}}\chi_T\chi_{T'}
    % = \int_{E_{\mu, \mu'}}\sum_{T\in\T_\lambda}\chi_T\sum_{T'\in\T'_{\lambda'}}\chi_{T'}.
\end{equation}

我们首先导出一个 $\lambda$ 的下界估计. 由 \eqref{eq: lambda exist},
\begin{align*}
    \mu\mu'\abs{E_{\mu, \mu'}}&\approx\int_{E_{\mu, \mu'}}\sum_{T\in\T_\lambda}\chi_T\sum_{T'\in\T'_{\lambda'}}\chi_T'\ls\int_{E_\mu}\mu'\sum_{T\in\T_\lambda}\chi_T\\
    &\sim\mu'\sum_{T\in\T_\lambda}\abs{T\cap E_{\mu, \mu'}}\sim\mu'\lambda\delta^{n-1}\#\T_\lambda\ls\mu'\lambda.
\end{align*}
从而
\begin{equation}
    \label{eq: lambda lower bound}
    \lambda\gtrapprox\mu\abs{E_{\mu, \mu'}}.
\end{equation}

然后我们尝试利用 \eqref{eq: lambda exist} 构造一个 ``高重数'' 的圆柱体.
\[
    \sum_{T\in\T_\lambda}\sum_{T'\in\T'_{\lambda'}}\int_{E_{\mu, \mu'}}\chi_T\chi_{T'} = \sum_{T\in\T_\lambda}\sum_{T'\in\T'_{\lambda'}}\abs{E\cap T\cap T'}\leq\sum_{T\in\T_\lambda}\sum_{T'\in\T'_{\lambda'}}\abs{T\cap T'},
\]
并且 $\#\T'_{\lambda'}\ls\delta^{1-n}$, 故由 \eqref{eq: lambda exist} 可知存在 $T'\in\T'_{\lambda'}$ 使得
\[
    \sum_{T\in\T_{\lambda}}\abs{T\cap T'}\gtrapprox\mu\mu'\abs{E_{\mu, \mu'}}\delta^{n-1}.
\]
而 $\angle(T, T')\sim 1$, 从而 $\abs{T\cap T'}\ls\delta^{n}$, 由此可知存在 $T'\in\T'_{\lambda'}$ 使得
\begin{equation}
    \label{eq: high-multi tube exist}
    \delta^n\#\set{T\in\T_\lambda: T\cap T'\neq\emptyset}\gtrapprox\delta^{n-1}\mu\mu'\abs{E_{\mu, \mu'}}.
.\end{equation}
至此, 我们就找到了一个 $T'\in\T'_{\lambda'}$, 其 ``重数'' 满足下界估计 \eqref{eq: high-multi tube exist}. 接下来我们将尝试用此 $T'$ 为 ``刷柄'', 尝试施展证明命题 \ref{prop: finite field hairbrush} 的论证方法.

\subsubsection{利用 ``刷子'' 导出估计}

取上一小节中构造的 $\mu, \mu', \lambda, \lambda'\in 2^{\Z}$, 以及 ``刷柄'' $T'\in\T'_{\lambda'}$. 记 $\mb{H}=\set{T\in\T_\lambda: T\cap T'\neq\emptyset}$ 为 ``刷毛'' 的集合. 由 \eqref{eq: high-multi tube exist} 可知
\begin{equation}
    \label{eq: num of hairbrush}
    \#\mb{H}\gtrapprox\delta^{-1}\mu\mu'\abs{E_{\mu, \mu'}}.
\end{equation}
我们希望证明 \eqref{eq: Dyadic partition, mu} 在 $q'=(n+2)/2$ 时的情形, 即证明
\begin{equation}
    \label{eq: hairbrush to prove}
    (\mu\mu')^{(n+2)/2n}\abs{E_{\mu, \mu'}}\lessapprox\delta^{-(n-2)/n}
\end{equation}

我们希望取适当的 $C$, 去掉使得 $T'$ 的邻域 $N=\set{x\in\R^n:\dist(x, T')<C^{-1}\lambda}$ 之后 $\mb{H}$ 中的圆柱体重叠程度比较低. 注意到 $N$ 可以被 $O(\lambda\delta C^{-1})$ 个平行于 $T'$ 的 $1\times\delta$ 圆柱体 $\set{\tau}$ 所覆盖, 于是对任意 $T\in\mb{H}$ 都有
\[
    \int_{E_{\mu, \mu'}}\chi_{T\cap N}\leq\sum_{\tau}\abs{\tau\cap T}\ls\lambda\delta C^{-1}\delta^n;
\]
另一方面
\[
    \int_{E_{\mu, \mu'}}\chi_T = \abs{T\cap E_{\mu, \mu'}} \sim\lambda\delta^{n-1},
\]
令 $C$ 充分大即可使得对任意 $T\in\mb{H}$ 都有
\[
    \int_{E_{\mu, \mu'}}\chi_{T\sm N}\gs\lambda\delta^{n-1}.
\]
对 $T$ 求和, 就得到 $\set{T\sm N}$ 平均重数的下界估计
\begin{equation}
    \label{eq: avg multi of hairs}
    \int_{E_{\mu, \mu'}}\sum_{T\in\mb{H}}\chi_{T\sm N}\gs\lambda\delta^{n-1}\#\mb{H}.
\end{equation}

类比命题 \ref{prop: finite field hairbrush} 的证明中的证明, 我们希望引入一些平面的估计来限制 $\mb{H}$ 的重叠. 于是我们尝试引入 $L^2$ 范数并应用 C\'ordoba 的估计. 由 \eqref{eq: avg multi of hairs} 以及 Cauchy-Schwarz 不等式可知,
\begin{equation}
    \label{eq: hairbrush L2 esti. lower}
    \norm{\sum_{T\in\mb{H}}\chi_{T\sm N}}_2\gs\lambda\delta^{n-1}\#\mb{H}\abs{E_{\mu, \mu'}}^{-1/2}.
\end{equation}
然后我们再尝试估计 $\norm{\sum_{T\in\mb{H}}\chi_{T\sm N}}_2$ 的上界, 将其平方并做二进制分割:
\begin{equation}
    \label{eq: hairbrush L2 esti.}
    \begin{split}
    \norm{\sum_{T\in\mb{H}}\chi_{T\sm N}}_2^2 &= \sum_{T_1, T_2\in H}\abs{T_1\cap T_2\cap N^c}\\
    & = \sum_{T_1\in\mb{H}}\abs{T_1\cap N^c} + \sum_{T_1\in\mb{H}}\sum_{k=0}^{O(\log 1/\delta)}\sum_{T_2\in\mb{H}: \angle(T_2, T_1)\sim 2^{-k}}\abs{T_1\cap T_2\cap N^c}.
    \end{split}
\end{equation}
上面式子中的第一项 $\ls\#\mb{H}\delta^{n-1}$; 用 C\'ordoba 的估计 \eqref{eq: Cordoba Geo Esti.} 来控制第二项:
\begin{equation}
    \label{eq: hairbrush cordoba dyadic}
    \begin{split}
    \sum_{T_2\in\mb{H}: \angle(T_2, T_1)\sim 2^{k}}&\abs{T_1\cap T_2\cap N^c}\\
    &\ls 2^{-k}\delta^n\#\set{T_2\in\mb{H}: \angle(T_1, T_2)\sim 2^{-k}, T_1\cap T_2\cap N^c\neq\emptyset}.
    \end{split}
\end{equation}
然后需要用到一个几何上的事实: 圆柱体组成的 ``三角形'' 一定会落在平面的某个邻域内.

\begin{lemma}[Wolff]
    若 $T_1, T_2$ 与 $T'$ 相交, 并且 $\angle(T_1, T')\sim\angle(T_2, T')\sim 1$. 若 $T_1\cap T_2\cap N^c\neq\emptyset$, 并且 $\angle(T_1, T_2)\sim 2^{-k}$, 则 $T_2$ 落在 $T'$ 和 $T_1$ 的主轴确定的平面的 $O(\delta/\lambda)$ 邻域内.
\end{lemma}

限于篇幅, 我们略去上述引理的证明. 由此引理可以得到
\begin{equation*}
    2^k\delta^n\#\set{T_2\in\mb{H}: \angle(T_1, T_2)\sim 2^{-k}, T_1\cap T_2\cap N^c\neq\emptyset}
    \ls\delta^{n-1}\lambda^{2-n},
\end{equation*}
代入 \eqref{eq: hairbrush cordoba dyadic} 和 \eqref{eq: hairbrush L2 esti.} 即可得到
\[
    \norm{\sum_{T\in\mb{H}}\chi_{T\sm N}}_2^2\ls\#\mb{H}\delta^{n-1}\lambda^{2-n}.
\]
再结合 \eqref{eq: hairbrush L2 esti. lower} 和 \eqref{eq: num of hairbrush}, 就得到
\[
    \mu\mu'\lambda^n\delta^{n-2}\lessapprox 1.
\]
然后代入 \eqref{eq: lambda lower bound}, 就得到
\[
    \mu^{n+1}\mu'\abs{E_{\mu, \mu'}}^n\lessapprox\delta^{2-n}.
\]
由 $\mu$ 和 $\mu'$ 的对称性, 故也有
\[
    \mu'^{n+1}\mu\abs{E_{\mu, \mu'}}^n\lessapprox\delta^{2-n}.
\]
以上两个式子相乘, 再开 $2n$ 次方, 即可得到 \eqref{eq: hairbrush to prove}. 至此就完成了定理 \ref{thm: Wolff} 的证明.

\subsection{Dvir 的多项式方法}

作为本节的结束, 我们介绍 Dvir 完全解决有限域 Kakeya 问题的多项式方法\cite{dvir2009size}. 但他的方法过于依赖有限域的性质, 无法如前文中 Bourgain 和 Wolff 的论证一样很好的推广到欧氏空间上.

\begin{theorem}[Dvir]
    \label{thm: Dvir}
    设 $\F$ 是 $q$ 元有限域. 若 $K\subset\F^n$ 为 Kakeya 集, 则
    \[
        \# K\gs_n q^n.
    \]
\end{theorem}

多项式方法依赖于如下基本的观察: 若任意 $d$ 次多项式 $f$ 都可被其在集合 $E\subset\F$ 上的取值完全确定, 即 $f\vert_E\equiv 0\implies f\equiv 0$, 则 $\# E\geq d+1$. 也就是说, 通过那些在 $E$ 上恒为零的多项式, 我们可以得出 $E$ 元素个数的估计. 事实上, 这种方法也可以推广到高维情形.

\begin{lemma}
    \label{lemma: 多项式方法}
    假如 $E\subset\F^n$ 满足: 对任意 $f\in\F[x_1, \cdots, x_n]$ s.t. $\deg f\leq d$, 若 $f\vert_E = 0$ 则 $f=0$; 那么 $\# E\geq\binom{n+d}{n}$.
\end{lemma}

\begin{proof}
    记 $\F[x_1, \cdots, x_n]_{\leq d}$ 为域 $\mb{F}$ 上不超过 $d$ 次的 $n$ 元多项式的集合. 考虑 $\F$-线性映射
    \[
        \Phi: \F[x_1, \cdots, x_n]_{\leq d}\to \F^E,\quad f\mapsto (f(x))_{x\in E}.
    \]
    由题设条件可知 $\ker\Phi=0$, 即 $\Phi$ 为单射. 而由组合数学的知识可知
    \[
        \dim \F[x_1, \cdots, x_n]_{\leq d} = \binom{n+d}{n},
    \]
    故 $\# E\geq\binom{n+d}{n}$.
\end{proof}

引理 \ref{lemma: 多项式方法} 的证明虽然很简单, 但它给了我们一种估计 $\F^n$ 中集合元素个数的方法 --- 考察次数多高的多项式能够被其在这个集合上的取值所完全确定.

\begin{proof}[\pftitle 定理 \ref{thm: Dvir} 的证明]
设 $K\subset\F^n$ 为 Kakeya 集. 由引理 \ref{lemma: 多项式方法} 可知, 我们只需证明:
\begin{equation}
    \label{eq: Dvir-poly-reduction}
    \text{对任意 $f\in\F[x_1, \cdots, x_n]$ s.t. $\deg f\leq q-1$, 若 $f\vert_K=0$, 则 $f=0$.}
\end{equation}
因为由引理 \ref{lemma: 多项式方法} 可知 \eqref{eq: Dvir-poly-reduction} 蕴含着
\[
    \# K\geq\binom{n+q-1}{n}\sim_n q^n.
\]

下面我们证明 \eqref{eq: Dvir-poly-reduction}. 由 Kakeya 集的性质, 任取 $v\in\F^n\sm\set{0}$, 都存在 $x\in\F^n$ 使得 $\set{x+tv: t\in\F}\subset K$, 从而 $g(t) := f(x+tv)$ 恒为零. 记 $f_i$ 为 $f$ 中所有 $i$ 次项的和 ($1\leq i\leq d$), 由于 $g(t)$ 是 $d$ 次多项式, 故 $g(t)$ 的 $d$ 次项系数为 $f_d(v)$, 于是 $f_d(v)=0\ \forall v\in\F^n\sm\set{0}$. 而 $f_d$ 齐 $d$ 次多项式, 从而 $f_d$ 在 $\F^n$ 上恒为零. 重复上述步骤, 就可以依次说明 $f_{d-1}=0$, $f_{d-2}=0$, 等等. 于是 $f_i=0\ \forall i$, 从而 $f=0$. 如此就完成了证明.
\end{proof}

%%%%%%%%%%%%%%%%%%%%%%%%%%%%%%%%%%%%%%%%%%%%%%%%%%%%%%%%%%%%%%%%

\section{Kakeya 问题和 Fourier 分析的联系}

Kakeya 问题和组合、数论、调和分析等多个领域中的多个重要问题有广泛的联系. 本节我们着重于 Fourier 分析, 简要介绍 Kakeya 问题和 Fourier 求和问题以及限制性估计问题之间的联系.

\subsection{球乘子的 \texorpdfstring{$L^p$}{L\^{}p} 无界性}

对 $f\in\mc{S}(\R^n)$, 定义 Fourier 部分和算子
\[
    S_Rf(x) := \int_{\abs{\xi}\leq R} \hat{f}(\xi)e^{2\pi i\xi\cdot x}\d\xi.
\]
由 Fourier 逆转公式可知在每一点 $x\in\R^n$ 处都有 $S_Rf(x)\to f(x)\ (R\to\infty)$. 接下来我们希望考察 $\set{S_Rf}$ 的 $L^p$ 收敛性, 即 $\norm{S_Rf-f}_p\to 0$ 是否成立.

由一致有界原理可知, $\norm{S_Rf-f}_p\to 0\ (\forall f\in\mc{S}(\R^n))$ 的充要条件是存在与 $R$ 无关的常数 $C$ 使得
\[
    \norm{S_Rf}_p\leq C\norm{f}_p\quad\forall f\in\mc{S}(\R^n).
\]
伸缩变换之后就等价于
\begin{equation*}
    % \label{eq: 球乘子有界性}
    \norm{S_1f}_p\leq C\norm{f}_p,\quad\forall f\in\mc{S}(\R^n).
\end{equation*}
在 $n=1$ 时, $S_1$ 就是 Hilbert 变换, 从而它在 $L^p\ (1<p<\infty)$ 上是有界的. 在 $n\geq 2$ 的情形, 由 Plancherel 定理可知 $S_1$ 在 $L^2$ 上有界; 但对 $p\neq 2$ 的 $L^p$ 空间, Fefferman 在 1970 年证明了如下出人意料的结果:

\begin{theorem}[Fefferman\cite{fefferman1971multiplier}]
    \label{thm: Fefferman Disk Multiplier}
    若 $n\geq 2$, $1<p<\infty$, 则 $S_1$ 在 $L^p$ 上有界当且仅当 $p=2$.
\end{theorem}

Fefferman 证明定理 \ref{thm: Fefferman Disk Multiplier} 的方法依赖于 Besicovitch 给出的零测度 Kakeya 集的构造, 这也是 Kakeya 问题在 Fourier 分析中最早的应用之一. 下面我们介绍 Fefferman 的证明方法.

首先, 由下面的引理 (为行文简洁我们略去证明), 我们只需证明 $n=2$ 时 $S_1$ 的无界性.

\begin{lemma}[De Leeuw]
    \label{lemma: De Leeuw}
    设 $m\in L^\infty(\R^n)$. 定义 $\R^n$ 上的 Fourier 乘子
    \[
        \widehat{Tf} := m\hat{f}\quad\forall f\in\mc{S}(\R^n)
    \]
    以及 $\R^{n-1}$ 上的 Fourier 乘子
    \[
        \widehat{T_0f}(\xi') := m(\xi', 0)\hat{f}(\xi')\quad\forall f\in\mc{S}(\R^{n-1}),\ \forall \xi'\in\R^{n-1}.
    \]
    那么, $T$ 在 $L^p(\R^n)$ 上有界蕴含 $T_0$ 在 $L^p(\R^{n-1})$ 上有界.
\end{lemma}

为了否定 $S_1$ 的有界性, 我们需要估计 $S_1f = f * \chi_{B_1}^\vee$ 的下界. 我们尝试用半平面 $H=\set{(\xi_1,\xi_2)\in\R^2: \xi_1>0}$ 来逼近 $B_1$, 因为 $\chi_H^\vee$ 比 $\chi_{B_1}^\vee$ 有更简单的表达式.

\begin{lemma}
    \label{lemma: 球乘子vs半平面乘子}
    若 $S_1$ 在 $L^p(\R^2)$ 上有界, 则 $S_Hf := (\chi_H\hat{f})^{\vee}$ 也在 $L^p(\R^2)$ 上有界. 
\end{lemma}

\begin{proof}
    注意到当 $R\to\infty$ 时, 半平面 $H$ 可以用圆盘 $B(R, R)$ 来逼近, 因此我们考虑平移之后的球乘子 $S_R'f := (\chi_{B(R, R)}\hat{f})^{\vee}$. 因为 $S_1$ 在 $L^p$ 上有界, 由前面的讨论可知 $\set{S_R}$ 在 $L^p$ 上一致有界, 从而 $\set{S_R'}$ 也在 $L^p$ 上一致有界. 不难验证 $\norm{S_R'f-S_Hf}_p\to 0\ \forall f\in C_c^\infty$, 从而 $S_H$ 也在 $L^p(\R^2)$ 上有界. 
\end{proof}

我们接下来只需要说明 $S_H$ 在 $L^p(\R^2)\ (p>2)$ 上是无界的. 在分布的意义下
\[
    \chi_H^{\vee}(x_1, x_2) = \delta_0(x_2)\,\text{sgn}^\vee(x_1) = \frac{i}{\pi}\delta_0(x_2)\,\text{p.v.}\left(\frac{1}{x_1}\right),
\]
于是
\begin{equation}
    \label{eq: 半平面乘子}
    S_H f (x_1, x_2) = (f * \chi_H^\vee) (x_1, x_2) = \frac{i}{\pi}\lim_{\e\to 0}\int_{\abs{x_1-y}>\e}f(y, x_2)\,\frac{dy}{x_1 - y}.
\end{equation}
由 \eqref{eq: 半平面乘子} 即可得出:

\begin{lemma}
    设 $\delta>0$, $T=[0, 1]\times [0, \delta]$. 若 $f\in C_c^\infty(T)$ 是一个截断函数, 满足 $0\leq f\leq 1$ 并且 $f$ 在 $[1/3, 2/3]\times [\delta/3, 2\delta/3]$ 上恒为 1, 则在将 $T$ 沿 $x_1$ 轴平移 10 个单位后得到的矩形 $T'$ 上, 有 $\abs{S_Hf}\gs 1$.
\end{lemma}

由此我们得出一个构造反例的思路: 选取一族 $1\times\delta$ 矩形 $\set{T}$, 在每个 $T$ 上取截断函数 $f_T$. 考虑函数 $f=\sum_{T} f_T$. 如果我们能够使得 $\set{T'}$ 的重合程度很高, 而 $S_H f_T$ 在 $T'$ 上比较大, 那么 $\norm{S_Hf}_p = \norm{\sum_{T}S_Hf_T}_p$ 就会比较大; 然后我们可以使得 $\set{T}$ 的重合程度很低 (因为 $T'$ 和 $T$ 之间的距离很大), 从而 $\norm{f}_p$ 就很小. 这时 Besicovitch 的构造就派上用场了, 它告诉我们的结论是:

\begin{lemma}
    \label{lemma: Besicovitch-Fefferman}
    对任意 $M>0$, 存在 $\delta>0$ 以及有限个\emph{两两不交的} $1\times\delta$ 矩形 $\set{T}$, 使得
    \[
        \abs{\bigcup_{T} T'}\leq M^{-1}\abs{\bigcup_{T} T}.
    \]
\end{lemma}

然后我们还需要解决一个问题: 在计算 $S_Hf = \sum_{T}S_Hf_T$ 的 $L^p$ 范数的时候, $S_H f_T$ 之间有可能存在很多的正负抵消, 导致 $\norm{S_Hf_T}_p$ 不够大. 我们可以通过 "随机化" 的技巧来估计正负抵消的影响. 通过概率论中的 Khintchine 不等式, 我们可以刻画正负抵消的平均程度.

\begin{lemma}[Khintchine 不等式]
    设 $\set{\e_T}$ 是独立的以 1/2 概率分别取 $+1$ 和 $-1$ 的随机变量序列. 那么对任意 $f_T\in L^p(\R^n)\ (0<p<\infty)$, 有
    \[
        \mb{E}\norm{\sum_{T}\e_Tf_T}_p^p \sim_p \norm{\left(\sum_T\abs{f_T}^2\right)^{1/2}}_p^p.
    \]
\end{lemma}

有了上述的所有准备之后, 我们就可以得出定理 \ref{thm: Fefferman Disk Multiplier} 的证明.

\begin{proof}[\pftitle 定理 \ref{thm: Fefferman Disk Multiplier} 的证明]
由 $S_H$ 的自伴性以及引理 \ref{lemma: 球乘子vs半平面乘子} 和引理 \ref{lemma: De Leeuw}, 我们只需证明 $2<p<\infty$ 时 $S_H$ 在 $L^p(\R^2)$ 上无界.

沿用前面的记号. 考虑 $f=\sum_T\e_T f_T$, 那么由 Khintchine 不等式可知
\[
    \mb{E}\norm{S_Hf}_p^p \sim_p \norm{\left(\sum_T\abs{S_Hf_T}^2\right)^{1/2}}_p^p.
\]
由于在 $T'$ 上 $\abs{S_Hf_T}\gs 1$, 于是
\[
    \norm{\left(\sum_T\abs{S_Hf_T}^2\right)^{1/2}}_p^p\gs\norm{\left(\sum_T\chi_{T'}\right)^{1/2}}_p^p = \norm{\sum_T\chi_{T'}}_{p/2}^{p/2}.
\]
由 H\"older 不等式以及引理 \ref{lemma: Besicovitch-Fefferman},
\[
    \norm{\sum_T\chi_{T'}}_1\leq\norm{\sum_T\chi_{T'}}_{p/2}\abs{\bigcup_T T'}^{1-2/p}\leq M^{2/p-1}\abs{\bigcup_T T}.
\]
由 $\set{T}$ 两两无交, 我们有
\begin{align*}
    \norm{\sum_T\chi_{T'}}_1 &= \sum_T\abs{T'} = \sum_T\abs{T} = \abs{\bigcup_T T},\\
    \norm{f}_p^p &= \sum_T\norm{f_T}_p^p.
\end{align*}
综合以上各式, 化简之后可得
\[
    \mb{E}\norm{S_Hf}_p^p\gs M^{p/2-1}\norm{f}_p^p.
\]

由期望的性质, 必然存在 $\set{\e_T}$ 的某种取值, 使得 $\norm{S_Hf}_p^p$ 的值不小于其期望. 由此我们就构造出了一个函数 $f$ 使得
\[
    \norm{S_Hf}_p\gs M^{1/2-1/p}\norm{f}_p.
\]
由 $M$ 的任意性即可得到当 $p>2$ 时 $S_H$ 在 $L^p$ 上无界. 
\end{proof}

Fefferman 给出的构造事实上是非常弱的, 只需对球乘子 $S_1$ 稍微做一些磨光就可以使得他的反例失效. 定义 \textbf{Bochner-Riesz 算子}为
\[
    \widehat{S_1^\e f}(\xi) = (1-\abs{\xi})^\e\chi_{B(0, 1)}\hat{f}(\xi).
\]
算子 $S_1^\e$ 的 $L^p$ 有界性至今仍然是公开问题.

\begin{conjecture}[Bochner-Riesz]
    对任意 $\e>0$ 和 $2n/(n+1)<p<2n/(n-1)$, 算子 $S_1^\e$ 在 $L^p$ 上是有界的. 
\end{conjecture}

Fefferman 的反例无法使得算子 $S_1^\e$ 无界, 而关于 Kakeya 集的进一步形式也许可以帮助构造出更强的反例.
从 Fefferman 的构造可以得出 Bochner-Riesz 猜想蕴含 Kakeya 猜想. 而在 1991 年 Bourgain 的一篇论文中给出了一种从 Kakeya 猜想的结果推导出 Bochner-Riesz 猜想的部分结果的论证方法.\cite{bourga1991besicovitch}

\subsection{限制性猜想和 Kakeya 猜想的关系}

\textbf{Fourier 限制性问题}指的是: 对 $f\in L^p(\R^n)\ (1\leq p\leq 2)$, 其 Fourier 变换 $\hat{f}\in L^{p'}(\R^n)$ 在集合 $S\subset\R^n$ 上的限制是否有意义? 例如当 $p=1$ 时, $\hat{f}$ 是连续函数, 因而定义其在任意集合 $S$ 上的限制; 但当 $p=2$ 时, $\hat{f}\in L^2(\R^n)$ 是在几乎处处的意义下定义的, 这时 $\hat{f}$ 在 Lebesgue 零测集 $S$ 上的限制就不一定有明确的意义. 

设 $S$ 是 $\R^n$ 中的超曲面, $d\sigma$ 是其表面测度. 对 $p, q\in [1, \infty]$, 若
\begin{equation}
    \label{eq: 限制性估计}
    \nnorm{\hat{f}}_{L^q(S,\, d\sigma)}\ls\norm{f}_{L^p(\R^n)}\quad\forall f\in\mc{S}(\R^n),
\end{equation}
那么算子 $R: f\in\mc{S}(\R^n)\mapsto\hat{f}\vert_S$ 可以唯一地延拓为 $L^p(\R^n)\to L^q(S,\, d\sigma)$ 的有界线性算子, 从而可以把 $R$ 看成是将 $L^p$ 函数的 Fourier 变换 ``限制'' 为 $S$ 上的 $L^p$ 函数. 也就是说, 解决 Fourier 限制性问题本质上是寻找形如 \eqref{eq: 限制性估计} 的估计; 我们称形如 \eqref{eq: 限制性估计} 的估计为 \textbf{限制性估计}. 

利用对偶性可知, 式 \eqref{eq: 限制性估计} 等价于
\begin{equation}
    \label{eq: 对偶限制性估计}
    \nnorm{\widehat{gd\sigma}}_{L^{p'}(\R^n)}\ls\norm{g}_{L^{q'}(S^{n-1})}\quad\forall g\in L^{\infty}(S^{n-1}), 
\end{equation}
其中
\[
    \widehat{gd\sigma}(\xi) = \int_{S^{n-1}}g(x)e^{-2\pi i\xi\cdot x}\d\sigma(x).
\]
式 \eqref{eq: 对偶限制性估计} 也称为\textbf{限制性估计}.

在 $S=S^{n-1}$ 为球面的情形, Stein 提出了如下猜想:

\begin{conjecture}[球面限制性猜想]
    若 $q>2n/(n-1)$, $q\geq (n+1)p'/(n-1)$, $p\geq 1$,\footnote{这些指标的条件是限制性估计成立的必要条件.} 则
    \begin{equation*}
        \nnorm{\widehat{gd\sigma}}_{L^{q}(\R^n)}\ls\norm{g}_{L^{p}(S^{n-1})}\quad\forall g\in L^\infty(S^{n-1}). 
    \end{equation*}
\end{conjecture}

事实上, 限制性估计 \eqref{eq: 限制性估计} 可以导出 Kakeya 极大算子的估计, 这个发现最早来自 Beckner 等人\cite{beckner1989note}. 

\begin{proposition}
    设 $q\geq 2n/(n-1)$. 那么, 限制性估计
    \label{prop: Restriction imply Kakeya}
    \begin{equation}
        \label{eq: 带 epsilon 的限制性估计}
        \nnorm{\widehat{gd\sigma}}_{q+\e}\ls_\e\norm{g}_{p}\quad\forall g\in L^\infty(S^{n-1})
    \end{equation}
    蕴含 Kakeya 极大算子的估计
    \begin{equation}
        \label{eq: Kakeya induced by Restriction}
        \norm{\sum_{T\in\T}\chi_T}_{q/2}\ls_\e\delta^{2(2n/q-(n-1)-\e)}.
    \end{equation}
    其中 $\T$ 是轴向 $\delta$-分离的一族 $1\times\delta$ 圆柱体.
\end{proposition}

下面开始证明 \ref{prop: Restriction imply Kakeya}. 首先注意到, 由 H\"older 不等式可知, 式 \eqref{eq: 带 epsilon 的限制性估计} 蕴含
\begin{equation}
    \label{eq: 局部限制性估计}
    \nnorm{\widehat{gd\sigma}}_{L^q(B(0, R))}\ls_\e R^\e\norm{g}_{L^p(S^{n-1})}\quad\forall g\in L^\infty(S^{n-1}).
\end{equation}
通常称 \eqref{eq: 局部限制性估计} 为\textbf{指数为 $\e$ 的局部限制性估计}. 然后注意到如下的引理, 它可以看成是不确定性原理的某种形式:

\begin{lemma}
    \label{lemma: Knapp example}
    对 $\omega\in S^{n-1}$, 定义 $R\times\sqrt{R}$ 圆柱体
    \[
        T_\omega := \set{\xi\in\R^n:\abs{\xi\cdot\omega}\leq R, \abs{x-(x\cdot\omega)\omega}\leq\sqrt{R}}
    \]
    若 $f_\omega$ 是球冠 $B(\omega, 1/\sqrt{100R})\cap S^{n-1}$ 的特征函数, 则
    \[
        \abs{\widehat{fd\sigma}(\xi)}\sim R^{-(n-1)/2}\quad\forall\xi\in T_\omega.
    \]
\end{lemma}

\begin{proof}
    注意到当 $\xi\in T_\omega$ 以及 $x\in B(\omega, 1/\sqrt{100R})$ 时, $\abs{\xi\cdot x-\xi\cdot\omega}\leq 1/10$, 故
    \begin{align*}
        \abs{\widehat{f_\omega d\sigma}(\xi)} &= \abs{\int_{\abs{x-\omega}<1/\sqrt{100R}}e^{-2\pi i\xi\cdot x}\d\sigma(x)}\\
        &\sim\abs{\int_{\abs{x-\omega}<1/\sqrt{100R}}e^{-2\pi i\xi\cdot \omega}\d\sigma(x)}\\
        &\sim R^{-(n-1)/2}.
    \end{align*}
    这就完成了证明.
\end{proof}

在上述引理中, 函数 $e^{2\pi ix\cdot h}f_\omega(x)$ 在平移后的圆柱体 $T_\omega-h$ 上的绝对值 $\sim R^{-(n-1)/2}$. 由此, 对任意极大的轴向 $(1/\sqrt{100R})$-分离的 $R\times\sqrt{R}$ 圆柱体族 $\T=\set{T}$, 可以找到函数 $f_T: S^{n-1}\to\C$ 使得
\begin{align*}
    &\spt f_T\subset B(\omega_T, 1/\sqrt{100R}),\quad\abs{f_T}=1;\\
    &\abs{\widehat{f_Td\sigma}(\xi)}\sim R^{-(n-1)/2}\quad\forall\xi\in T.
\end{align*}
当 $\set{T}$ 的重叠程度很高时, 和式 $\sum_T\widehat{f_Td\sigma}$ 中就会有很多正负抵消的项. 然后我们就可以采用标准的随机化技巧来估计抵消的程度.

\begin{proof}[\pftitle 命题 \ref{prop: Restriction imply Kakeya} 的证明]
取 $\set{\e_T}$ 为独立地以 $1/2$ 概率分别取 $\pm 1$ 的随机变量序列, 考虑函数 $f=\sum_T\e_Tf_T$. 不妨设任意 $T$ 都包含于 $B(0, 100R)$, 则由 Khintchine 不等式可知
\begin{equation*}
    \mb{E}\nnorm{\widehat{fd\sigma}}_{L^q(B(0, 100R))}^q\sim\norm{\left(\sum_T\abs{\widehat{f_Td\sigma}}^2\right)^{1/2}}_{L^q(B(0, 100R))}^q.
\end{equation*}
由于 $\widehat{f_Td\sigma}$ 在 $T$ 上的绝对值 $\sim R^{-(n-1)/2}$, 故
\[
    \mb{E}\nnorm{\widehat{fd\sigma}}_q^q\gs\norm{\left(\sum_T R^{-(n-1)}\abs{\chi_T}^2\right)^{1/2}}_q^q = R^{-(n-1)q/2}\norm{\sum_T\chi_T}_{q/2}^{q/2}.
\]
另一方面, $\set{f_T}$ 的支集几乎不重叠, 可以验证
\[
    \norm{f}_p\sim\left(\sum_T\norm{f_T}_p^p\right)^{1/p}\sim\left(R^{-(n-1)/2}\#\T\right)^{1/p}\sim 1.
\]
结合以上两式, 利用局部限制性估计 \eqref{eq: 局部限制性估计}, 即可得到
\[
    \norm{\sum_T\chi_T}_{q/2}\ls_\e R^{n-1+\e}.
\]

我们上面得出的估计离 \eqref{eq: Kakeya induced by Restriction} 一些差距. 经观察之后考虑尺度变换. 记 $\tilde{T}=RT$ 用 $\tilde{T}$ 代替 $T$ 重复上述讨论, 可得
\[
    \norm{\sum_T\chi_{\tilde{T}}}_{q/2}\ls_\e R^{-2n/q+n-1+\e}.
\]
然后再把 $R^2\times R$ 伸缩为 $1\times\delta$ 即可得到 \eqref{eq: Kakeya induced by Restriction}. ($\delta=R^{-1/2}$)
\end{proof}

由命题 \ref{prop: Restriction imply Kakeya} 不难看出, 球面限制性猜想蕴含 Kakeya 极大算子猜想, 但至今仍然不知道反过来的结论是否正确. 在 1991 年 Bourgain 证明了反方向的部分结果.

\begin{theorem}[Bourgain\cite{bourga1991besicovitch}]
    如果有 Kakeya 极大算子的估计
    \[
        \norm{\sum_{T\in\T}\chi_T}_{p'}\ls_\e\delta^{-n/p+1-\e},
    \]
    其中 $\T$ 是轴向 $\delta$-分离的一族 $1\times\delta$ 圆柱体; 那么当
    \[
        q > 2\left(\frac{p'}{n+1}+\frac{n}{n-1}\right)
    \]
    时, 有限制性估计
    \[
        \nnorm{\widehat{fd\sigma}}_p\ls\norm{f}_\infty\quad\forall f\in L^\infty(S^{n-1}).
    \]
\end{theorem}

在 Bourgain 1991 年的论文\cite{bourga1991besicovitch}中, 他正是首先利用 bush 论证得出了当时最优的 Kakeya 极大算子的估计, 然后由此推动了限制性猜想的进展. 

\bibliography{ref.bib}
\bibliographystyle{plain}

\end{document}
